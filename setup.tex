\documentclass{vkr}
\usepackage[english, russian]{babel} % переносы
\usepackage{graphicx} % для вставки картинок
\graphicspath{{images/}} % путь к изображениям
\usepackage[hidelinks]{hyperref}

\usepackage{xltabular} % для вставки таблиц
\usepackage{makecell}
\renewcommand\theadfont{} % шрифт в /thead
\usepackage{array} % для определения новых типов столбцов таблиц
\newcolumntype{T}{>{\centering\arraybackslash}X} % новый тип столбца T - автоматическая ширина столбца с выравниванием по центру
\newcolumntype{R}{>{\raggedleft\arraybackslash}X} % новый тип столбца R - автоматическая ширина столбца с выравниванием по правому краю
\newcolumntype{C}[1]{>{\centering\let\newline\\\arraybackslash\hspace{0pt}}m{#1}} % новый тип столбца C - фиксированная ширина столбца с выравниванием по центру
\newcolumntype{r}[1]{>{\raggedleft\arraybackslash}p{#1}} % новый тип столбца r - фиксированная ширина столбца с выравниванием по правому краю
\newcommand{\centrow}{\centering\arraybackslash} % командой \centrow можно центрировать одну ячейку (заголовок) в столбце типа X или p, оставив в оcтальных ячейках другой тип выравнивания
\newcommand{\finishhead}{\endhead\hline\endlastfoot}
\newcommand{\continuecaption}[1]{\caption*{#1}\\ \hline }
\usepackage{etoolbox}
\AtBeginEnvironment{xltabular}{\refstepcounter{tablecnt}} % подсчет таблиц xltabular, обычные таблицы подсчитываются в классе

% из-за того что babel переопределяет имена заголовков
\addto\captionsrussian{\renewcommand{\figurename}{Рисунок}}
