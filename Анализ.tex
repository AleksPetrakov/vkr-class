\section{Анализ предметной области}

\subsection{Исследование предметной области}

В эпоху цифровых трансформаций и динамичных социальных изменений, сфера услуг для домашних животных испытывает значительные инновационные сдвиги. Современные тенденции ориентированы на продвижение здорового образа жизни домашних животных, что стимулирует разработку и внедрение новаторских услуг и продуктов. Особенно актуальным становится индивидуальный подход к заботе о каждом питомце, что предполагает использование интегрированных и многофункциональных решений для удовлетворения потребностей как животных, так и их хозяев.

Индустрия услуг, включая ветеринарную помощь, груминг и передержку, активно интегрирует современные технологии и методики. Примечательно, что интерес к телемедицине в сфере ветеринарии растет, обеспечивая удаленные консультации специалистов. Кроме того, роль информационных ресурсов для обучения и поддержки владельцев животных значительно возрастает, что способствует улучшению качества ухода за питомцами.

Расширение ассортимента товаров для домашних животных, включая корма, одежду, аксессуары и игрушки, также заслуживает внимания. Прогресс в сфере электронной коммерции и смарт-технологий, например, умные кормушки и трекеры активности, усиливает динамику роста этой отрасли.

Таким образом, рынок услуг и товаров для домашних животных преобразуется в сложный и разносторонний сектор, который требует инновационных и комплексных подходов для удовлетворения потребностей животных и их владельцев. В этом контексте, цифровые платформы и технологии играют ключевую роль, предоставляя удобство, доступность и высокое качество услуг и товаров.

\subsection{Роль цифровых платформ}

Цифровая революция оказала значительное влияние на рынок услуг для домашних животных, особенно в сегменте временного содержания животных, который демонстрирует устойчивый рост. Широкий спектр услуг, от традиционных гостиниц для животных до услуг индивидуальных нянь, теперь доступен благодаря цифровизации. Это обеспечивает удобство бронирования, управление услугами, а также повышает прозрачность и доступность предложений. Кроме того, возрастающее внимание к экологии и этике в уходе за животными становится заметным.

Цифровые платформы, предоставляющие услуги для домашних животных, предлагают широкий диапазон онлайн-сервисов – от информационных ресурсов до комплексных приложений для бронирования и управления услугами. Они значительно упрощают процесс поиска, бронирования и оплаты услуг, включая:

\begin{itemize}
  \item Страхование домашних животных;
  \item Выездной груминг;
  \item Выгул собак;
  \item Услуги няни для котов;
  \item Консультации психолога для котов;
  \item Гостиницы для животных;
  \item Зоотакси;
  \item Фитнес и тренировки для собак;
  \item Телемедицина для животных.
\end{itemize}

Среди преимуществ таких решений - удобство доступа, способность к быстрому сравнению цен и услуг, а также возможность ознакомления с отзывами других пользователей. Однако, стоит отметить и некоторые недостатки, такие как ограниченная персонализация услуг для конкретных потребностей животных, риски в области безопасности данных и зависимость от качества интернет-соединения.

Для успешной работы таких платформ крайне важно разработать многофункциональную модель данных, включающую информацию о пользователях (владельцах животных), поставщиках услуг, доступных услугах, ценах, расписании и отзывах. Это обеспечит эффективное взаимодействие между пользователями и поставщиками, повышая уровень удовлетворенности клиентов.

<<Социомаркет для "владельцев домашних животных">> обладает рядом уникальных функций, предназначенных для удовлетворения потребностей владельцев домашних животных. Основные функции включают создание, редактирование, поиск и просмотр карточек с товарами и услугами. Это позволяет пользователям в полной мере управлять информацией о предлагаемых услугах, таких как:

\begin{itemize}
  \item Выездной груминг;
  \item Выгул собак;
  \item Временная передержка собак;
  \item Услуги няни для котов;
  \item Продажа животных.
\end{itemize}

Цифровые платформы, такие как <<Петбук>> и <<Петмир>>, выполняют ключевую роль в соединении владельцев домашних животных с поставщиками услуг. Они обеспечивают пользователей инструментами для сравнения цен, просмотра отзывов и участия в онлайн-сообществах. Разнообразие бизнес-моделей этих платформ, включая модели прямых продаж и подписок, способствует их эффективной монетизации.

\subsection{Технологии разработки веб-приложений}

Разработка серверной части веб-приложений включает в себя использование множества технологий и подходов. Среди основных можно выделить:

\begin{itemize}
  \item \textbf{SQL-ориентированные базы данных:} Эти базы данных, включая MySQL, PostgreSQL и Microsoft SQL Server, предпочтительны для систем с необходимостью сложных запросов и транзакционной целостности.
  \item \textbf{NoSQL базы данных,} включая MongoDB и Cassandra, используются для более гибкой структуры данных и лучшей масштабируемости.
\end{itemize}

Выбор .NET Core и Angular для разработки был обусловлен их гибкостью, масштабируемостью и удобством в сопровождении.
PostgreSQL выбран в качестве СУБД, т.к. позволяет эффективно обрабатывать большие объемы данных и сложные запросы, а также за поддержку JSON, что делает PostgreSQL идеальной СУБД для гибких веб-приложений.

\subsection{Анализ аудитории пользователей веб-приложений для владельцев домашних животных}

Аудитория веб-приложений для владельцев домашних животных представляет собой разнообразную группу, объединенную общей заботой о благополучии своих питомцев. Эта группа включает людей различных возрастных категорий, социально-экономических статусов и образовательных уровней, но с общими интересами в области ухода за животными и использования технологий для улучшения качества жизни своих питомцев.

\textbf{Демографические Характеристики:}
Большинство пользователей веб-приложений для домашних животных принадлежат к возрастной группе от 25 до 45 лет, активно используют цифровые технологии в повседневной жизни и часто ищут удобные и инновационные способы улучшения ухода за своими животными.

\textbf{Поведенческие Особенности:}
Пользователи часто ищут информацию о лучших практиках ухода за животными, сравнивают различные продукты и услуги, исследуют отзывы других пользователей и активно участвуют в онлайн-сообществах. Они предпочитают платформы, которые предлагают простой и интуитивно понятный интерфейс, а также ценят возможность получить персонализированные рекомендации.

\textbf{Потребности и Предпочтения:}
Важными факторами для пользователей являются надежность, безопасность и доступность услуг. Они ожидают высокого качества обслуживания, прозрачности цен и легкости доступа к услугам. Предпочтения включают разнообразие предлагаемых услуг, таких как онлайн-консультации ветеринаров, автоматизированные напоминания о медицинских процедурах и индивидуально подобранные программы питания и ухода за животными.

\textbf{Влияние на Дизайн и Функциональность Веб-Приложения:}
На основе анализа аудитории, разработка веб-приложения должна быть сфокусирована на создании удобного и функционального интерфейса, который облегчает навигацию и поиск информации. Важным аспектом является интеграция функций обратной связи и коммуникационных инструментов, позволяющих пользователям делиться опытом и получать поддержку. Внедрение аналитических инструментов для сбора данных о предпочтениях пользователей поможет в создании более персонализированных и целенаправленных предложений, тем самым повышая удовлетворенность и лояльность клиентов.

В заключение, аудитория веб-приложений для владельцев домашних животных требует комплексного подхода, который сочетает в себе удобство, информативность и возможность персонализации. Понимание их потребностей и предпочтений является ключом к разработке успешного и востребованного

\subsection{Перспективы развития}

В перспективе развития веб-приложения ключевым является масштабирование его возможностей. Для Angular-приложений это может включать применение ленивой загрузки модулей, оптимизацию производительности и улучшение пользовательского интерфейса. .NET приложения могут масштабироваться путем интеграции микросервисов, что позволяет разделить приложение на независимые компоненты, упрощая управление и обновление. В будущем можно рассмотреть интеграцию бэкенда в виде набора микросервисов, а фронтенда – как микрофронтенда, что обеспечит более гибкое управление ресурсами и улучшит масштабируемость приложения.

Сектор ветеринарной медицины и ухода за домашними животными переживает значительные трансформации благодаря технологическим инновациям. Прогресс в этой сфере включает использование передовых диагностических инструментов, роботизированных хирургических систем и искусственного интеллекта для диагностики, что позволяет повысить точность и эффективность медицинской помощи. Особое внимание уделяется генетическим тестам для определения наследственных заболеваний и разработке новых методов лечения, включая терапию стволовыми клетками и плазматерапию. Важной является интеграция ИИ для повышения точности диагностики и эффективности лечения в ветеринарной медицине.
