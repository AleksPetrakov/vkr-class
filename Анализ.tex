\section{Анализ предметной области}

\subsection{Исследование предметной области}

Современный мир цифровых трансформаций и социальных изменений вносит заметные инновации в сферу услуг для домашних животных \cite{freedman}. Наблюдается укрепление трендов, направленных на улучшение качества жизни домашних питомцев, что стимулирует появление новых услуг и продуктов. Особенно важным становится принцип индивидуального подхода к уходу за животными, подразумевающий применение интегрированных решений, учитывающих потребности как животных, так и их владельцев.

Ветеринарные услуги, груминг и передержка животных активно адаптируют новейшие технологии и методологии. В частности, расширение сферы телемедицины в ветеринарии позволяет владельцам получать консультации специалистов на расстоянии. Важным становится также предоставление информационных ресурсов для поддержки и обучения владельцев животных, улучшая тем самым общий уровень ухода за питомцами.

Рынок товаров для домашних животных, включающий корма, одежду, аксессуары и игрушки, также испытывает расширение. Развитие электронной торговли и смарт-технологий, например, интеллектуальных кормушек и трекеров активности, способствует усилению динамики роста в этой области \cite{market}.

Таким образом, рынок услуг и товаров для домашних животных превращается в сложную и разнообразную сферу, требующую инновационных подходов для удовлетворения потребностей животных и их владельцев. Цифровые платформы и технологии занимают ключевую роль, обеспечивая удобство, доступность и качество услуг и продуктов.

\subsection{Роль цифровых платформ}

Применение цифровых технологий приводит к существенным изменениям в сегменте услуг для домашних животных, особенно в области временного содержания животных \cite{freedman}. Разнообразие предлагаемых услуг, от гостиниц для животных до индивидуальных услуг по уходу, становится доступным благодаря цифровой интеграции. Это упрощает процессы бронирования и управления услугами, увеличивая их доступность и прозрачность. Важность экологической устойчивости и этических стандартов в уходе за животными также возрастает.

Цифровые платформы предлагают широкий спектр онлайн-сервисов, от информационных порталов до комплексных приложений для бронирования и управления услугами. Это значительно упрощает поиск, бронирование и оплату услуг, включая:

\begin{itemize}
  \item страхование домашних животных;
  \item выездной груминг;
  \item выгул собак;
  \item услуги няни для котов;
  \item консультации психолога для котов;
  \item гостиницы для животных;
  \item зоотакси;
  \item фитнес и тренировки для собак;
  \item телемедицина для животных.
\end{itemize}

Преимущества этих решений включают удобство доступа, возможность быстрого сравнения цен и услуг, а также получение обратной связи от пользователей. Однако существуют и определенные ограничения, такие как необходимость персонализации услуг в соответствии с индивидуальными потребностями животных, а также вопросы безопасности данных и зависимости от качества интернет-соединения \cite{market}.

Для эффективной работы цифровых платформ критично важно создать многофункциональную модель данных, охватывающую информацию о пользователях, поставщиках услуг, доступных услугах, ценах, расписании и отзывах. Это улучшит взаимодействие между пользователями и поставщиками и повысит уровень удовлетворенности клиентов \cite{freedman}.

Исходя из этих данных, веб-приложение <<Социомаркет для владельцев домашних животных>> должно обладать рядом уникальных функций, предназначенных для удовлетворения потребностей владельцев домашних животных. Основные функции включают создание, редактирование, поиск и просмотр карточек с товарами и услугами. Это позволяет пользователям в полной мере управлять информацией о предлагаемых услугах, таких как:

\begin{itemize}
  \item выездной груминг;
  \item выгул собак;
  \item временная передержка собак;
  \item услуги няни для котов;
  \item продажа животных.
\end{itemize}

В настоящее время в Российской Федерации не так широко развит рынок цифровых платформ для владельцев домашних животных. Тем не менее на рынке представлены цифровые платформы, такие как <<Петбук>> и <<Петмир>>. Они выполняют ключевую роль в соединении владельцев домашних животных с поставщиками услуг. Они обеспечивают пользователей инструментами для сравнения цен, просмотра отзывов и участия в онлайн-сообществах. Разнообразие бизнес-моделей этих платформ, включая модели прямых продаж и подписок, способствует их эффективной монетизации \cite{market}.

\subsection{Технологии разработки веб-приложений}

Разработка серверной части веб-приложений включает в себя использование множества технологий и подходов. Среди основных можно выделить:

\begin{itemize}
  \item SQL-ориентированные базы данных -\- Эти базы данных, включая MySQL, PostgreSQL и Microsoft SQL Server, предпочтительны для систем с необходимостью сложных запросов и транзакционной целостности \cite{freedman};
  \item NoSQL базы данных -\- включая MongoDB и Cassandra, используются для более гибкой структуры данных и лучшей масштабируемости \cite{market}.
\end{itemize}

Выбор .NET Core в сочетании с Entity Framework был обусловлен их гибкостью, масштабируемостью и удобством в сопровождении. .NET Core предоставляет широкие возможности для модульного тестирования контроллеров, моделей и других модулей программной системы.

PostgreSQL была выбрана в качестве SQL-ориентированной СУБД, т.к. позволяет эффективно обрабатывать большие объемы данных и сложные запросы, а также за поддержку JSON, что делает PostgreSQL идеальной СУБД для гибких веб-приложений \cite{freedman}.

Выбор Angular для разработки клиетской части приложения обусловлен его способностью к расширяемости и простоте модуляризации. Применение Agular компонентов для разработки пользоветельского интерфейса также способствует простоте кода и его повторному использованию (реиспользование компонентов).

\subsection{Анализ аудитории пользователей веб-приложений для владельцев домашних животных}

Аудитория веб-приложений для владельцев домашних животных представляет собой разнообразную группу, объединенную общей заботой о благополучии своих питомцев. Эта группа включает людей различных возрастных категорий, социально-экономических статусов и образовательных уровней, но с общими интересами в области ухода за животными и использования технологий для улучшения качества жизни своих питомцев \cite{market}.

Большинство пользователей веб-приложений для домашних животных принадлежат к возрастной группе от 25 до 45 лет, активно используют цифровые технологии в повседневной жизни и часто ищут удобные и инновационные способы улучшения ухода за своими животными \cite{freedman}.

Пользователи часто ищут информацию о лучших практиках ухода за животными, сравнивают различные продукты и услуги, исследуют отзывы других пользователей и активно участвуют в онлайн-сообществах. Они предпочитают платформы, которые предлагают простой и интуитивно понятный интерфейс, а также ценят возможность получить персонализированные рекомендации \cite{market}.

Важными факторами для пользователей являются надежность, безопасность и доступность услуг. Они ожидают высокого качества обслуживания, прозрачности цен и легкости доступа к услугам. Предпочтения включают разнообразие предлагаемых услуг, таких как онлайн-консультации ветеринаров, автоматизированные напоминания о медицинских процедурах и индивидуально подобранные программы питания и ухода за животными \cite{freedman}.

На основе анализа аудитории, разработка веб-приложения должна быть сфокусирована на создании удобного и функционального интерфейса, который облегчает навигацию и поиск информации. Важным аспектом является интеграция функций обратной связи и коммуникационных инструментов, позволяющих пользователям делиться опытом и получать поддержку. Внедрение аналитических инструментов для сбора данных о предпочтениях пользователей поможет в создании более персонализированных и целенаправленных предложений, тем самым повышая удовлетворенность и лояльность клиентов.

В заключение, аудитория веб-приложений для владельцев домашних животных требует комплексного подхода, который сочетает в себе удобство, информативность и возможность персонализации. Понимание их потребностей и предпочтений является ключом к разработке успешного и востребованного

\subsection{Перспективы развития}

В перспективе развития веб-приложения ключевым является масштабирование его возможностей. Для Angular-приложений это может включать применение ленивой загрузки модулей, оптимизацию производительности и улучшение пользовательского интерфейса.

Angular-приложения имеют возможность серверного рендеринга для обеспечения наиболее эффективной индексации поисковыми роботами. Такое дополнительно улучшение доступно для реализации при помощи подключаемой библиотеки AngularUniversal. 

.NET приложения могут масштабироваться путем интеграции микросервисов, что позволяет разделить приложение на независимые компоненты, упрощая управление и обновление. В будущем можно рассмотреть интеграцию бэкенда в виде набора микросервисов, а фронтенда – как микрофронтенда, что обеспечит более гибкое управление ресурсами и улучшит масштабируемость приложения.

Сектор ветеринарной медицины и ухода за домашними животными переживает значительные трансформации благодаря технологическим инновациям. Прогресс в этой сфере включает использование передовых диагностических инструментов, роботизированных хирургических систем и применения искусственного интеллекта для диагностики, что позволяет повысить точность и эффективность медицинской помощи в ветеринарии.

Для расширения функций веб-приложения следует уделить особое внимание идеям и технологиям, применяемым в инновационных проектов в медицине.

Благодаря генетическим тестам для определения наследственных заболеваний и разработке новых методов лечения, включая терапию стволовыми клетками и плазматерапию, врачи могут выявлять предрасположенность к различным заболеваниям у людей и животных с последующим полным исцелением от болезни.

Важной является интеграция ИИ для повышения точности диагностики и эффективности лечения в ветеринарной медицине, а не только в классической медицине.
