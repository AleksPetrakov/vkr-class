\section*{ВВЕДЕНИЕ}
\addcontentsline{toc}{section}{ВВЕДЕНИЕ}

Цифровизация и инновации в сфере услуг для домашних животных значительно расширяют горизонты для предпринимателей и пользователей. Настоящая дипломная работа посвящена разработке веб-приложения «Социомаркет для владельцев домашних животных», направленного на улучшение доступности и качества услуг для владельцев домашних животных.

\emph{Цель настоящей работы} -\- разработка веб-приложения, объединяющего владельцев домашних животных и предоставляющих услуги для них. 

Для достижения цели настоящей работы были поставлены \emph{следующие задачи:}

\begin{itemize}
    \item провести анализ предметной области;
    \item разработать пользовательский интерфейс клиентской части приложения;
    \item спроектировать архитектуру и разработать серверную часть приложения;
    \item реализовать клиент-серверное взаимодействие частей приложений.
\end{itemize}

\emph{Структура и объем работы.} Данный отчет представляет собой выпускную квалификационную работу, которая включает в себя введение, четыре основных раздела, заключение, список использованных источников и два приложения. Общий объем текста этой работы составляет \formbytotal{page}{страниц}{у}{ы}{}.

\emph{Во введении} определена цель и поставлены задачи, описана структура работы и представлено краткое содержание каждого из разделов.

\emph{В первом разделе} подробно рассмотрены актуальные аспекты рынка услуг для животных, рассмотрено воздействие цифровой трансформации на обслуживание, а также проанализированы инновационные тенденции и технологии, связанные с предоставлением услуг временного содержания животных.

\emph{Во втором разделе} сформулированы основные требования к разработке веб-приложения. В нем содержатся цели и задачи проекта, а также требования пользователя к интерфейсу. Кроме того, данный раздел охватывает различные функциональные аспекты и обсуждает архитектуру системы.

\emph{В третьем разделе} объясняется выбор используемых технологий проектирования, таких как .NET Core, Entity Framework Core, PostgreSQL и Angular. Также в этом разделе осуществляется проектирование пользовательского интерфейса и предоставление описания классов системы.

\emph{В четвертом разделе} данной работы представлен подробный список классов и их методов, которые были использованы в процессе разработки сайта. Кроме того, в этом разделе проводится детальное тестирование разработанного веб-сайта, с целью проверки его функциональности и стабильности.

\emph{В заключении} преподносятся основные выводы и результаты, полученные в ходе разработки проекта.

В приложении А представлен графический материал.
В приложении Б представлены фрагменты исходного кода серверной части приложения. 
В приложении В представлены фрагменты исходного кода клиентской части приложения. 