\appendix{Фрагменты исходного кода серверной части приложения}

Program.cs
\lstset{style=sharpc}
\begin{lstlisting}
    using Microsoft.AspNetCore.Hosting;
    using Microsoft.Extensions.Hosting;
    
    namespace new_back
    {
      public static class Program
      {
        public static void Main(string[] args)
        {
          CreateHostBuilder(args).Build().Run();
        }
    
        private static IHostBuilder CreateHostBuilder(string[] args) =>
            Host.CreateDefaultBuilder(args)
                .ConfigureWebHostDefaults(webBuilder =>
                {
                  webBuilder.UseStartup<Startup>();
                });
      }
    }
\end{lstlisting}

Startup.cs
\lstset{style=sharpc}
\begin{lstlisting}
    using Microsoft.AspNetCore.Builder;
    using Microsoft.AspNetCore.Hosting; // Убедитесь, что используете это пространство имен
    using Microsoft.Extensions.Configuration;
    using Microsoft.Extensions.DependencyInjection;
    using Microsoft.EntityFrameworkCore;
    using new_back.Infrastructure;
    
    namespace new_back
    {
        public class Startup
        {
            public Startup(IConfiguration configuration)
            {
                Configuration = configuration;
            }
    
            private IConfiguration Configuration { get; }
    
            public void ConfigureServices(IServiceCollection services)
            {
                services.AddDbContext<DatabaseContext>(options =>
                    options.UseNpgsql(Configuration.GetConnectionString("new_backConnection"))
                );
    
                // добавляем поддержку CORS:
                services.AddCors();
    
                services.AddMvc(option => option.EnableEndpointRouting = false);
    
                services.AddSwaggerGen();
                services.AddScoped<ProductRepository>();
                services.AddScoped<ProductGroupRepository>();
                services.AddScoped<UserRepository>();
                services.AddScoped<UserRoleRepository>();
                services.AddScoped<CityRepository>();
            }
    
            public void Configure(IApplicationBuilder app, IWebHostEnvironment env) // Используйте IWebHostEnvironment здесь
            {
                if (env.IsDevelopment())
                {
                    app.UseDeveloperExceptionPage();
    
                    app.UseSwagger();
    
                    app.UseSwaggerUI(c =>
                    {
                        c.SwaggerEndpoint("/swagger/v1/swagger.json", "Animal Sharing API version 0.1");
                    });
                }
                else
                {
                    app.UseHsts();
                }
    
                // строчки, позволяющие серверу обрабатывать статические файлы:
                app.UseDefaultFiles();
                app.UseStaticFiles();
    
                // строка, позволяющая серверу принимать запросы с порта 4200:
                app.UseCors(builder => builder.WithOrigins("http://localhost:4200"));
    
                app.UseMvc();
            }
        }
    }
\end{lstlisting}

Short.cs
\lstset{style=sharpc}
\begin{lstlisting}
    namespace new_back.Models

    {
      public abstract class Short
      {
        public long Id { get; set; }
    
        public string? Name { get; set; }
    
        public string? Description { get; set; } = null;
      }
    }
\end{lstlisting}

City.cs
\lstset{style=sharpc}
\begin{lstlisting}
    namespace new_back.Models
    {
        public class City : Short
        {
        }
    }
\end{lstlisting}

HashSalt.cs
\lstset{style=sharpc}
\begin{lstlisting}
    namespace new_back.Models
    {
        public class HashSalt
        {
            public string Hash { get; set; }
            public string Salt { get; set; }
        }
    }
\end{lstlisting}

Product.cs
\lstset{style=sharpc}
\begin{lstlisting}
    namespace new_back.Models

    {
      public class Product : Short
      {
        public string? ImageUrl { get; set; } // assuming JSON string
        public decimal Price { get; set; }
        public long? CreatedByUserId { get; set; }
        public long CityId { get; set; }
        public long ProductGroupId { get; set; }
        public long ProductStatusId { get; set; }
        public DateTime CreatedAt { get; set; }
        public DateTime? PublishedAt { get; set; }
        public DateTime? ExpiredAt { get; set; }
        public long PriorityId { get; set; }
        public DateTime PriorityStartedAt { get; set; }
        public DateTime? PriorityExpiredAt { get; set; }
      }
      
      public class ProductView : Product
      {
        public UserView? CreatedByUser { get; set; }
      }
    }
\end{lstlisting}

CityIdEnum.cs
\lstset{style=sharpc}
\begin{lstlisting}
    // CityEnum.cs in Enums folder
    using System.ComponentModel;
    
    namespace new_back.Enums
    {
        public enum CityIdEnum
        {
            [Description("Москва")]
            Moscow = 1,
            [Description("Санкт-Петербург")]
            SaintPetersburg = 2,
            [Description("Курск")]
            Kazan = 3,
        }
    }    
\end{lstlisting}

ProductTypeEnum.cs
\lstset{style=sharpc}
\begin{lstlisting}
    using System.ComponentModel;

    namespace new_back.Enums
    {
        public enum ProductTypeEnum
        {
            [Description("Собаки")]
            Dogs = 0,
            [Description("Кошки")]
            Cats = 1,
            [Description("Лошади")]
            Horses = 2,
            [Description("Хорьки")]
            Ferrets = 3,
            [Description("Рептилии")]
            Reptiles = 4,
            [Description("Грызуны")]
            Rodents = 5,
            [Description("Амфибии")]
            Amphibians = 6,
            [Description("Птицы")]
            Birds = 7,
            [Description("Экзотика")]
            Exotic = 8,
            [Description("Беспозвоночные")]
            Invertebrates = 9,
            [Description("Аквариумные рыбки")]
            AquariumFish = 10,
            [Description("Сельскохозяйственные животные")]
            FarmAnimals = 11,
            [Description("Другое")]
            Others = 12,
        }
    }
\end{lstlisting}

UserRoleEnum.cs
\lstset{style=sharpc}
\begin{lstlisting}
    using System.ComponentModel;

    namespace new_back.Enums
    {
        public enum UserRoleEnum
        {
            [Description("Ситтер")]
            SITTER = 1,
            [Description("Хозяин")]
            OWNER = 2,
        }
    }
\end{lstlisting}

ProductGroup.cs
\lstset{style=sharpc}
\begin{lstlisting}
    using System.Collections.Generic;
    using new_back.Enums;
    
    namespace new_back.Models
    {
        public class ProductGroup : Short
        {
            public ProductTypeEnum Type { get; set; }
            
            public List<long> ChildrenProductGroupIds { get; set; }
            
            public long ParentProductGroupId { get; set; }
        }
    }
\end{lstlisting}

ProductPriority.cs
\lstset{style=sharpc}
\begin{lstlisting}
    namespace new_back.Models
    {
        public class ProductPriority : Short
        {
            public decimal Price { get; set; }
            public int DaysCount { get; set; }
        }
    }
\end{lstlisting}

ProductStatus.cs
\lstset{style=sharpc}
\begin{lstlisting}
    namespace new_back.Models
    {
        public class ProductStatus : Short
        {
        }
    }
\end{lstlisting}

User.cs
\lstset{style=sharpc}
\begin{lstlisting}
    using new_back.Enums;

    namespace new_back.Models
    {
        public class UserBase
        {
            public string? Phone { get; set; }
            public string Email { get; set; }
            public string FirstName { get; set; }
            public string LastName { get; set; }
            public string? ImageUrl { get; set; }
            public string? Telegram { get; set; }
            public string? Website { get; set; }
            public string? UserRoleId { get; set; }
            public string? Permissions { get; set; }
        }
        
        public class UserRegistrationData : UserBase
        {
            public string Password { get; set; }
        }
    
        public class UserLoginData
        {
            public string Email { get; set; }
            public string Password { get; set; }
        }
        
        public class UserLoginResponseData : UserBase
        {
            public long Id { get; set; }
            public string AccessToken { get; set; }
        }
        
        public class UserView : UserBase
        {
            public long Id { get; set; }
        }
        
        public class User : UserBase
        {
            public long Id { get; set; }
            public string Hash { get; set; }
            public string Salt { get; set; }
        }
    }    
\end{lstlisting}

UserRole.cs
\lstset{style=sharpc}
\begin{lstlisting}
    namespace new_back.Models

    {
      public class UserRole : Short
      {
        public string? Permissions { get; set; }
      }
    }
\end{lstlisting}

CityController.cs
\lstset{style=sharpc}
\begin{lstlisting}
    using Microsoft.AspNetCore.Mvc;
    using new_back.Infrastructure;
    using new_back.Models;
    using System.Collections.Generic;
    using System.Linq;
    using System.Threading.Tasks;
    
    namespace new_back.Controllers
    {
        [ApiController]
        [Route("api/[controller]")]
        public class CityController : ControllerBase
        {
            private readonly CityRepository _cityRepository;
    
            public CityController(CityRepository cityRepository)
            {
                _cityRepository = cityRepository;
            }
    
            [HttpGet]
            public async Task<ActionResult<IEnumerable<City>>> GetAllCities()
            {
                var cities = await _cityRepository.GetAllAsync();
                return Ok(cities);
            }
        }
    }
\end{lstlisting}

ProductController.cs
\lstset{style=sharpc}
\begin{lstlisting}
    using Microsoft.AspNetCore.Mvc;
    using System;
    using System.Collections.Generic;
    using System.Linq;
    using new_back.Infrastructure;
    using new_back.Models;
    using new_back.Shared;
    
    namespace new_back.Controllersnew_back
    {
      [Route("api/[controller]")]
      [ApiController]
      public class ProductController : ControllerBase
      {
        private readonly ProductRepository _repository;
        private readonly UserRepository _userRepository;
    
        public ProductController(ProductRepository repository, UserRepository userRepository)
        {
          _repository = repository ?? throw new ArgumentNullException(nameof(repository));
          _userRepository = userRepository ?? throw new ArgumentNullException(nameof(userRepository));
        }
    
        private static ProductView MapProductToProductView(Product p, List<UserView> users)
        {
          return new ProductView
          {
            Id = p.Id,
            Name = p.Name,
            Description = p.Description,
            ImageUrl = p.ImageUrl,
            ProductGroupId = p.ProductGroupId,
            Price = p.Price,
            CreatedByUserId = p.CreatedByUserId,
            CreatedByUser = users.Find(u => u.Id == p.CreatedByUserId),
            CityId = p.CityId,
            PublishedAt = p.PublishedAt,
            ExpiredAt = p.ExpiredAt,
            ProductStatusId = p.ProductStatusId,
            CreatedAt = p.CreatedAt
          };
        }
        
        private static Product MapProductViewToProduct(ProductView p)
        {
          return new Product
          {
            Id = p.Id,
            Name = p.Name,
            Description = p.Description,
            ImageUrl = p.ImageUrl,
            ProductGroupId = p.ProductGroupId,
            Price = p.Price,
            CreatedByUserId = p.CreatedByUserId,
            
            CityId = p.CityId,
            PublishedAt = p.PublishedAt,
            ExpiredAt = p.ExpiredAt,
            ProductStatusId = p.ProductStatusId,
            CreatedAt = p.CreatedAt
          };
        }
    
        [HttpGet]
        public ActionResult<List<ProductView>> GetAll()
        {
          var users = _userRepository.GetAllUserViews();
          
          return _repository
            .GetAll()
            .Select(p => MapProductToProductView(p, users))
            .ToList();
        }
        
        [HttpGet("current-user")]
        public ActionResult<List<Product>> GetByUser()
        {
          try
          {
            var user = Token.GetAuthorizedUser(HttpContext, _userRepository);
            var users = _userRepository.GetAllUserViews();
            var items = _repository
              .FilterByUserId(user.Id)
              .Select(p => MapProductToProductView(p, users))
              .ToList();
            
            return Ok(items);
          }
          catch (Exception e)
          {
            return Problem(e.Message);
          }
        }
    
        [HttpGet("{id:long}")]
        public ActionResult<Product> GetById(long id)
        {
          var item = _repository.GetById(id >= 0 ? id : throw new ArgumentNullException(nameof(id)));
          if (item == null)
          {
            return NotFound();
          }
          return item;
        }
    
        [HttpPut("{id:long}")]
        public ActionResult<Product> Update(ProductView product, long id)
        {
          try
          {
            var user = Token.GetAuthorizedUser(HttpContext, _userRepository);
    
            if (product.CreatedByUserId != user.Id) throw new Exception("User does not have access to change this product");
            
            if (product is null) throw new ArgumentNullException(nameof(product));
            
            var dishDb = _repository.Update(MapProductViewToProduct(product));
            
            if (dishDb == null)
            {
              return NotFound();
            }
    
            return NoContent();
          }
          catch (Exception e)
          {
            return Problem(e.Message);
          }
        }
    
        [HttpPost]
        public ActionResult<Product> Create(ProductView productView)
        {
          try
          {
            if (productView is null) throw new ArgumentNullException(nameof(productView));
            var user = Token.GetAuthorizedUser(HttpContext, _userRepository);
    
            var product = MapProductViewToProduct(productView);
            product.CreatedByUserId = user.Id;
            
            var item = _repository.Create(product);
    
            return CreatedAtAction(
              nameof(GetById),
              new { id = item.Id },
              item
            );
          }
          catch (Exception e)
          {
            return Problem(e.Message);
          }
        }
    
        [HttpDelete("{id:long}")]
        public ActionResult Delete(long id)
        {
          try
          {
            var user = Token.GetAuthorizedUser(HttpContext, _userRepository);
            var oldProduct = _repository.GetById(id);
    
            if (oldProduct.CreatedByUserId != user.Id) throw new Exception("User does not have access to delete this product");
    
            var item = _repository.Delete(id >= 0 ? id : throw new ArgumentNullException(nameof(id)));
    
            if (item == null)
            {
              return NotFound();
            }
    
            return NoContent();
          }
          catch (Exception e)
          {
            return Problem(e.Message);
          }
        }
    
        [HttpGet("filter")]
        public ActionResult<List<Product>> GetFiltered(
          [FromQuery] long? cityId, 
          [FromQuery] long? productGroupId, 
          [FromQuery] DateTime? publishedAtFrom, 
          [FromQuery] DateTime? expiredAtTo
        )
        {
          try
          {
            var users = _userRepository.GetAllUserViews();
            var filteredProducts = _repository
              .GetFiltered(
                cityId, 
                productGroupId, 
                publishedAtFrom, 
                expiredAtTo
              )
              .Select(p => MapProductToProductView(p, users))
              .ToList();;
            
    
            // Assuming GetFiltered returns IEnumerable<Product>
            return Ok(filteredProducts);
          }
          catch (Exception e)
          {
            return Problem(e.Message);
          }
        }
      }
    }
\end{lstlisting}

ProductGroupController.cs
\lstset{style=sharpc}
\begin{lstlisting}
    using System;
    using System.Collections.Generic;
    using Microsoft.AspNetCore.Mvc;
    using new_back.Infrastructure;
    using new_back.Models;
    
    namespace new_back.Controllers
    {
        [Route("api/[controller]")]
        [ApiController]
        public class ProductGroupController : ControllerBase
        {
            private readonly ProductGroupRepository _repository;
            
            public ProductGroupController(ProductGroupRepository repository)
            {
                _repository = repository ?? throw new ArgumentNullException(nameof(repository));
            }
            
            [HttpGet]
            public ActionResult<List<ProductGroup>> GetAll()
            {
                return _repository.GetAll();
            }
            
            [HttpPost]
            public ActionResult<List<ProductGroup>> UpdateAll(List<ProductGroup> complexities)
            {
                var items = _repository.UpdateAll(
                    complexities ?? throw new ArgumentNullException("Ошибка при обновлении группы продуктов")
                );
    
                return CreatedAtAction(
                    nameof(GetAll),
                    null,
                    items
                );
            }
        }
    }
\end{lstlisting}

UserController.cs
\lstset{style=sharpc}
\begin{lstlisting}
    using System;
    using System.Security.Cryptography;
    using new_back.Infrastructure;
    using new_back.Models;
    using new_back.Shared;
    using Microsoft.AspNetCore.Mvc;
    
    
    namespace new_back.Controllers
    {
        [Route("api/[controller]")]
        [ApiController]
        public class UserController : ControllerBase
        {
            private readonly UserRepository _repository;
            
            public UserController(UserRepository repository)
            {
                _repository = repository ?? throw new ArgumentNullException(nameof(repository));
            }
            
            [HttpGet("{id:long}")]
            public ActionResult<UserView> GetById(long id)
            {
                var user = _repository.GetById(id >= 0 ? id : throw new ArgumentNullException(nameof(id)));
                
                if (user == null)
                {
                    return NotFound();
                }
                
                var userView = new UserView
                {
                    Id = user.Id,
                    Phone = user.Phone,
                    Email = user.Email,
                    FirstName = user.FirstName,
                    LastName = user.LastName,
                    ImageUrl = user.ImageUrl,
                    Telegram = user.Telegram,
                    Website = user.Website,
                    UserRoleId = user.UserRoleId,
                };
                
                return userView;
            }
            
            // [HttpPut("{id}")]
            // public ActionResult<User> Update(UserView user, string id)
            // {
            //     var dishDb = _repository.Update(user ?? throw new ArgumentNullException(nameof(user)));
            //     if (dishDb == null)
            //     {
            //         return NotFound();
            //     }
            //
            //     return NoContent();
            // }
    
            [HttpPost("create")]
            public ActionResult<User> Create(UserRegistrationData regUserData)
            {
                if (regUserData is null) throw new ArgumentNullException(nameof(regUserData));
    
                var hashSalt = HashPassword(regUserData.Password);
                var user = new User
                {
                    Id = 0,
                    Phone = regUserData.Phone,
                    Email = regUserData.Email,
                    FirstName = regUserData.FirstName,
                    LastName = regUserData.LastName,
                    ImageUrl = regUserData.ImageUrl,
                    Telegram = regUserData.Telegram,
                    Website = regUserData.Website,
                    Hash = hashSalt.Hash,
                    Salt = hashSalt.Salt,
                    UserRoleId = regUserData.UserRoleId,
                };
    
                var r = _repository.Create(user);
    
                if (r is null) return Problem("User with this email exists");
    
                var userWithToken = new UserLoginResponseData
                {
                    Id = r.Id,
                    Phone = r.Phone,
                    Email = r.Email,
                    FirstName = r.FirstName,
                    LastName = r.LastName,
                    ImageUrl = r.ImageUrl,
                    Telegram = r.Telegram,
                    Website = r.Website,
                    UserRoleId = r.UserRoleId,
                    AccessToken = Token.CreateToken(user)
                };
                
                return CreatedAtAction(
                    nameof(GetById),
                    new { id = userWithToken.Id },
                    userWithToken
                );
            }
            
            [HttpPost("login")]
            public ActionResult<UserLoginResponseData> Login(UserLoginData userData)
            {
                if (userData is null) throw new ArgumentNullException(nameof(userData));
            
                var r = _repository.GetByEmail(userData.Email);
                
                if (r is null) return Problem("There is no user with this email");
    
                var isPasswordMatched = VerifyPassword(userData.Password, r.Hash, r.Salt);
                
                if (!isPasswordMatched) return Problem("Incorrect password");
            
                var userWithToken = new UserLoginResponseData
                {
                    Id = r.Id,
                    Phone = r.Phone,
                    Email = r.Email,
                    FirstName = r.FirstName,
                    LastName = r.LastName,
                    ImageUrl = r.ImageUrl,
                    Telegram = r.Telegram,
                    Website = r.Website,
                    UserRoleId = r.UserRoleId,
                    AccessToken = Token.CreateToken(r)
                };
                
                return CreatedAtAction(
                    nameof(GetById),
                    new { id = userWithToken.Id },
                    userWithToken
                );
            }
            
            private static HashSalt HashPassword(string password)
            {
                var saltBytes = new byte[64];
                var provider = new RNGCryptoServiceProvider();
                provider.GetNonZeroBytes(saltBytes);
                var salt = Convert.ToBase64String(saltBytes);
                var rfc2898DeriveBytes = new Rfc2898DeriveBytes(password, saltBytes, 10000);
                var hashPassword = Convert.ToBase64String(rfc2898DeriveBytes.GetBytes(256));
    
                return new HashSalt { Hash = hashPassword, Salt = salt };
            }
            
            private static bool VerifyPassword(string enteredPassword, string storedHash, string storedSalt)
            {
                var saltBytes = Convert.FromBase64String(storedSalt);
                var rfc2898DeriveBytes = new Rfc2898DeriveBytes(enteredPassword, saltBytes, 10000);
                
                return Convert.ToBase64String(rfc2898DeriveBytes.GetBytes(256)) == storedHash;
            }
        }
    }    
\end{lstlisting}

UserRoleController.cs
\lstset{style=sharpc}
\begin{lstlisting}
    using System;
    using System.Collections.Generic;
    using Microsoft.AspNetCore.Mvc;
    using new_back.Infrastructure;
    using new_back.Models;
    
    namespace new_back.Controllers
    {
        [Route("api/[controller]")]
        [ApiController]
        public class UserRoleController : ControllerBase
        {
            private readonly UserRoleRepository _repository;
            
            public UserRoleController(UserRoleRepository repository)
            {
                _repository = repository ?? throw new ArgumentNullException(nameof(repository));
            }
            
            [HttpGet]
            public ActionResult<List<UserRole>> GetAll()
            {
                return _repository.GetAll();
            }
            
            [HttpPost]
            public ActionResult<List<UserRole>> UpdateAll(List<UserRole> complexities)
            {
                var items = _repository.UpdateAll(
                    complexities ?? throw new ArgumentNullException("Ошибка при обновлении роли пользователя")
                );
    
                return CreatedAtAction(
                    nameof(GetAll),
                    null,
                    items
                );
            }
        }
    }
\end{lstlisting}

CityRepository.cs
\lstset{style=sharpc}
\begin{lstlisting}
    using System;
    using System.Collections.Generic;
    using System.Linq;
    using Microsoft.EntityFrameworkCore;
    using new_back.Enums;
    using new_back.Models;
    using new_back.Shared;
    
    namespace new_back.Infrastructure
    {
        public class CityRepository
        {
            private readonly DatabaseContext _context;
    
            public CityRepository(DatabaseContext context)
            {
                _context = context;
            }
    
            private void EnsureDefaultCities()
            {
                if (!_context.Cities.Any())
                {
                    var cities = Enum.GetValues(typeof(CityIdEnum))
                        .Cast<Enum>()
                        .Select(e => new City { Name = e.GetDescription() })
                        .ToList();
    
                    _context.Cities.AddRange(cities);
                    _context.SaveChanges();
                }
            }
    
            public async Task<List<City>> GetAllAsync()
            {
                EnsureDefaultCities();
                return await _context.Cities.ToListAsync();
            }
        }
    }    
\end{lstlisting}

DatabaseBContext.cs
\lstset{style=sharpc}
\begin{lstlisting}
    using Microsoft.EntityFrameworkCore;
    using new_back.Models;
    
    namespace new_back.Infrastructure
    {  
      public class DatabaseContext : DbContext
      {
            internal object _context;
    
            // protected override void OnConfiguring(DbContextOptionsBuilder optionsBuilder)
            // {
            //     if (!optionsBuilder.IsConfigured)
            //     {
            //         optionsBuilder.UseSqlServer("Host=localhost;Port=5432;Username=postgres;Password=postgres;Database=AnimalSharing;");
            //     }
            // }
    
            public DatabaseContext() { }
    
        public DatabaseContext(DbContextOptions<DatabaseContext> options) : base(options)
        {
        }
    
        public DbSet<Product> Products { get; set; }
        public DbSet<ProductGroup> ProductGroups { get; set; }
        public DbSet<User> Users { get; set; }
        public DbSet<UserRole> UserRoles { get; set; }
        public DbSet<City> Cities { get; set; }
      }
    }
\end{lstlisting}

ProductGroupRepository.cs
\lstset{style=sharpc}
\begin{lstlisting}
    using System;
    using System.Collections.Generic;
    using System.Linq;
    using Microsoft.EntityFrameworkCore;
    using new_back.Enums;
    using new_back.Models;
    using new_back.Shared;
    
    namespace new_back.Infrastructure
    {
        public class ProductGroupRepository
        {
            private readonly DatabaseContext _context;
            
            public ProductGroupRepository(DatabaseContext context)
            {
                _context = context;
            }
    
            private void PushDefaultValues()
            {
                var typeNames = Enum.GetNames(typeof(ProductTypeEnum));
    
                for (int i = 0; i < typeNames.Length; i++)
                {
                    var enumValue = (ProductTypeEnum)Enum.Parse(typeof(ProductTypeEnum), typeNames[i]);
                    // Получаем описание из GetDescription()
                    string description = enumValue.GetDescription();
    
                    _context.Database.ExecuteSqlInterpolated(
                        $"INSERT INTO public.\"ProductGroups\"(\"Name\", \"Type\", \"ChildrenProductGroupIds\", \"ParentProductGroupId\") VALUES ({description}, {i}, {new List<long>(0)}, {0})"
                    );
                }
                
                _context.SaveChanges();
            }
            
            public List<ProductGroup> GetAll()
            {
                var productGroups = _context.ProductGroups
                    .FromSqlRaw("SELECT * FROM public.\"ProductGroups\" ORDER BY \"Type\" ASC")
                    .ToList();
    
                if (productGroups.Count == 0)
                {
                    PushDefaultValues();
                }
                
                return _context.ProductGroups.FromSqlRaw("SELECT * FROM public.\"ProductGroups\" ORDER BY \"Type\" ASC").ToList();
            }
            
            public List<ProductGroup> UpdateAll(List<ProductGroup> productGroups)
            {
                foreach (var productGroup in productGroups)
                {
                    _context.Database.ExecuteSqlInterpolated(
                        $"UPDATE public.\"ProductGroups\" SET \"Name\"={productGroup.Name}, \"Description\"={productGroup.Description} Where \"Id\" = {productGroup.Id}"
                    );
                    
                    try
                    {
                        _context.SaveChanges();
                    }
                    catch (DbUpdateConcurrencyException) when (!_context.ProductGroups.IsExists(productGroup.Id))
                    {
                        return null;
                    }
                }
    
                var updated = GetAll();
    
                return updated;
            }
        }
    }
\end{lstlisting}

ProductRepository.cs
\lstset{style=sharpc}
\begin{lstlisting}
    using System.Collections.Generic;
    using System.Linq;
    using Microsoft.EntityFrameworkCore;
    using new_back.Models;
    using new_back.Shared;
    
    namespace new_back.Infrastructure
    {
      public class ProductRepository
      {
        private readonly DatabaseContext _context;
    
        public ProductRepository(DatabaseContext context)
        {
          _context = context;
        }
    
        public List<Product> GetAll()
        {
          return _context.Products.ToList();
        }
    
        public List<Product> FilterByUserId(long id)
        {
          return _context.Products.FromSqlRaw("SELECT * FROM public.\"Products\" ORDER BY \"Id\" ASC").ToList();
        }
    
        public Product GetById(long id)
        {
          var item = _context.Products.Find(id);
    
          return item;
        }
    
        public Product Update(Product product)
        {
          _context.Products.Update(product);
    
          var updated = GetById(product.Id);
    
          try
          {
            _context.SaveChanges();
          }
          catch (DbUpdateConcurrencyException) when (!_context.Products.IsExists(product.Id))
          {
            return null;
          }
    
          return updated;
        }
    
        public Product Create(Product product)
        {
          _context.Products.Add(product);
          _context.SaveChanges();
    
          return product;
        }
    
        public Product Delete(long id)
        {
          var createdProduct = GetById(id);
    
          _context.Products.Remove(createdProduct);
          _context.SaveChanges();
    
          return createdProduct;
        }
    
        public IEnumerable<Product> GetFiltered(
          long? cityId, 
          long? productGroupId, 
          DateTime? publishedAtFrom, 
          DateTime? expiredAtTo)
        {
          var query = _context.Products.AsQueryable();
    
          if (cityId.HasValue)
            query = query.Where(p => p.CityId == cityId);
    
          if (productGroupId.HasValue)
            query = query.Where(p => p.ProductGroupId == productGroupId);
    
          // Подготовка отдельных запросов
          var queryA = publishedAtFrom.HasValue
              ? query.Where(p => p.PublishedAt >= publishedAtFrom.Value)
              : query;
    
          var queryB = expiredAtTo.HasValue
              ? query.Where(p => p.ExpiredAt <= expiredAtTo.Value)
              : query;
    
          // Объединение запросов
          var combinedQuery = queryA.Union(queryB);
    
          return query.ToList();
        }
      }
    }
\end{lstlisting}

UserRepository.cs
\lstset{style=sharpc}
\begin{lstlisting}
    using System.Collections.Generic;
    using System.Linq;
    using new_back.Models;
    using new_back.Shared;
    
    namespace new_back.Infrastructure
    {
        public class UserRepository
        {
            private readonly DatabaseContext _context;
            
            public UserRepository(DatabaseContext context)
            {
                _context = context;
            }
            
            public User GetById(long id)
            {
                var item = _context.Users?.Find(id);
    
                return item;
            }
            
            public User GetByEmail(string email)
            {
                var item = _context.Users?.FirstOrDefault(u => u.Email == email);
    
                return item;
            }
            
            public List<UserView> GetAllUserViews()
            {
                return _context.Users?.ToList().Select(u => new UserView
                {
                    Id = u.Id,
                    Phone = u.Phone,
                    Email = u.Email,
                    FirstName = u.FirstName,
                    LastName = u.LastName,
                    ImageUrl = u.ImageUrl,
                    Telegram = u.Telegram,
                    Website = u.Website,
                    UserRoleId = u.UserRoleId,
                }).ToList();
            }
            
            // public UserView Update(UserView user)
            // {
            //     _context.Users.Update(user);
            //
            //     var updated = GetById(user.Id);
            //
            //     try
            //     {
            //         _context.SaveChanges();
            //     }
            //     catch (DbUpdateConcurrencyException) when (!_context.Users.IsExists(user.Id))
            //     {
            //         return null;
            //     }
            //
            //     return updated;
            // }
    
            public User Create(User user)
            {
                if (_context.Users.IsUserEmailExists(user.Email)) return null;
                
                _context.Users.Add(user);
                _context.SaveChanges();
    
                return user;
            }
        }
    }
\end{lstlisting}

UserRoleRepository.cs
\lstset{style=sharpc}
\begin{lstlisting}
    using System;
    using System.Collections.Generic;
    using System.Linq;
    using Microsoft.EntityFrameworkCore;
    using new_back.Enums;
    using new_back.Models;
    using new_back.Shared;
    
    namespace new_back.Infrastructure
    {
        public class UserRoleRepository
        {
            private readonly DatabaseContext _context;
            
            public UserRoleRepository(DatabaseContext context)
            {
                _context = context;
            }
    
            private void PushDefaultValues()
            {
                var typeNames = Enum.GetNames(typeof(UserRoleEnum));
                var emptystr = "";
    
                for (int i = 0; i < typeNames.Length; i++)
                {
                    _context.Database.ExecuteSqlInterpolated(
                        $"INSERT INTO public.\"UserRoles\"(\"Name\", \"Permissions\") VALUES ({typeNames[i]}, {emptystr})"
                    );
                }
                
                _context.SaveChanges();
            }
            
            public List<UserRole> GetAll()
            {
                var productGroups = _context.UserRoles
                    .FromSqlRaw("SELECT * FROM public.\"UserRoles\" ORDER BY \"Id\" ASC")
                    .ToList();
    
                if (productGroups.Count == 0)
                {
                    PushDefaultValues();
                }
                
                return _context.UserRoles.FromSqlRaw("SELECT * FROM public.\"UserRoles\" ORDER BY \"Id\" ASC").ToList();
            }
            
            public List<UserRole> UpdateAll(List<UserRole> productGroups)
            {
                foreach (var UserRole in productGroups)
                {
                    _context.Database.ExecuteSqlInterpolated(
                        $"UPDATE public.\"UserRoles\" SET \"Name\"={UserRole.Name}, \"Description\"={UserRole.Description} Where \"Id\" = {UserRole.Id}"
                    );
                    
                    try
                    {
                        _context.SaveChanges();
                    }
                    catch (DbUpdateConcurrencyException) when (!_context.UserRoles.IsExists(UserRole.Id))
                    {
                        return null;
                    }
                }
    
                var updated = GetAll();
    
                return updated;
            }
        }
    }
\end{lstlisting}

EnumExtension.cs
\lstset{style=sharpc}
\begin{lstlisting}
    using System.ComponentModel;
    using System.Linq;
    using Microsoft.EntityFrameworkCore;
    using new_back.Models;
    
    namespace new_back.Shared
    {
        public static class EnumExtensions
        {
            public static string GetDescription(this Enum value)
            {
                    var field = value.GetType().GetField(value.ToString());
                    var attribute = Attribute.GetCustomAttribute(field, typeof(DescriptionAttribute)) as DescriptionAttribute;
                    return attribute?.Description ?? value.ToString();
            }
        }
    }
\end{lstlisting}

RepositoryHelper.cs
\lstset{style=sharpc}
\begin{lstlisting}
    using System.Linq;
    using Microsoft.EntityFrameworkCore;
    using new_back.Models;
    
    namespace new_back.Shared
    {
      public static class RepositoryHelper
      {
        public static bool IsExists<T>(this DbSet<T> dbSet, long id) where T : Short
        {
          return dbSet.Any(e => e.Id == id);
        }
        
        public static bool IsUserEmailExists<T>(this DbSet<T> dbSet, string email) where T : User
        {
          return dbSet.Any(e => e.Email == email);
        }
      }
    }
\end{lstlisting}

Token.cs
\lstset{style=sharpc}
\begin{lstlisting}
    using System;
    using System.Collections.Generic;
    using System.IO;
    using System.Linq;
    using System.Security.Claims;
    using System.Security.Cryptography;
    using new_back.Infrastructure;
    using new_back.Models;
    using Microsoft.AspNetCore.Http;
    using Newtonsoft.Json.Linq;
    using Org.BouncyCastle.Crypto;
    using Org.BouncyCastle.Crypto.Parameters;
    using Org.BouncyCastle.OpenSsl;
    using Org.BouncyCastle.Security;
    
    namespace new_back.Shared
    {
        public class Token
        {
            public static string CreateToken(User user)
            {
                // reading the content of a private key PEM file, PKCS8 encoded 
                var privateKey = System.IO.File.ReadAllText(@"/Users/aleks/openssl/privateKey.pem");
                
                RSA rsa = RSA.Create(); // Создание экземпляра RSA
    
                try
                {
                    rsa.ImportFromPem(privateKey); // Импорт ключа из PEM файла
    
                    var claims = new List<Claim>();
                    claims.Add(new Claim("id", user.Id.ToString()));
                    claims.Add(new Claim("email", user.Email));
    
                    Dictionary<string, object> payload = claims.ToDictionary(k => k.Type, v => (object)v.Value);
                    return Jose.JWT.Encode(payload, rsa, Jose.JwsAlgorithm.RS256);
                }
                finally
                {
                    rsa.Dispose(); // Убедитесь, что ресурсы освобождены
                }
            }
            
            public static JObject DecodeToken(string token)
            {
                var publicKey = System.IO.File.ReadAllText(@"/Users/aleks/openssl/publicKey.pem");
                
                RSA rsa = RSA.Create(); // Создание экземпляра RSA
    
                try
                {
                    rsa.ImportFromPem(publicKey); // Импорт ключа из PEM файла
    
                    // This will throw if the signature is invalid
                    var decoded = Jose.JWT.Decode(token, rsa, Jose.JwsAlgorithm.RS256);
                    return JObject.Parse(decoded);  
                }
                finally
                {
                    rsa.Dispose(); // Убедитесь, что ресурсы освобождены
                }
            }
    
            public static User GetAuthorizedUser(HttpContext context, UserRepository userRepository)
            {
                if (!context.Request.Headers.TryGetValue("Authorization", out  var authorizationToken))
                {
                    throw new Exception("Invalid credentials");
                }
          
                var jwt = authorizationToken.ToString().Replace("Bearer ", "");
                var decodedToken = Token.DecodeToken(jwt);
                var user = userRepository.GetById((long)decodedToken["id"]);
    
                if (user is null) throw new Exception("Invalid credentials");
            
                return user;
            }
        }
    }
\end{lstlisting}

\appendix{Фрагменты исходного кода клиентской части приложения}

app-routing.module.ts
\lstset{style=ts}
\begin{lstlisting}
    import { NgModule } from '@angular/core';
    import { RouterModule, Routes } from '@angular/router';
    import {SignupComponent} from "@app/pages/signup/signup.component";
    import {SignupClientComponent} from "@app/pages/signup/signup-client/signup-client.component";
    import {SignupSitterComponent} from "@app/pages/signup/signup-sitter/signup-sitter.component";
    import {SignupFormComponent} from "@app/pages/signup/signup-form/signup-form.component";
    import {SearchComponent} from "@app/pages/search/search.component";
    import {LoginComponent} from "@app/pages/login/login.component";
    import {SearchResultsComponent} from "@app/pages/search/search-results/search-results.component";
    import {UserComponent} from "@app/pages/user/user.component";
    
    const routes: Routes = [
      {
        path: '',
        pathMatch: 'full',
        redirectTo: 'product',
      },
      {
        path: 'product',
        loadChildren: () => import('./pages/products/product.module').then(m => m.ProductModule),
      },
      {
        path: 'signup',
        component: SignupComponent,
      },
    
      {
        path: 'signup-client',
        component: SignupClientComponent,
      },
      {
        path: 'signup-sitter',
        component: SignupSitterComponent,
      },
      {
        path: 'search',
        component: SearchComponent,
      },
      {
        path: 'signup-form',
        component: SignupFormComponent,
      },
      {
        path: 'login',
        component: LoginComponent,
      },
      { path: 'search-results',
        component: SearchResultsComponent,
      },
      { path: 'user',
        component: UserComponent,
      }
    ];
    
    @NgModule({
      imports: [RouterModule.forRoot(routes, { relativeLinkResolution: 'legacy' })],
      exports: [RouterModule],
    })
    export class AppRoutingModule {}
\end{lstlisting}

app.component.html
\begin{lstlisting}
    <mat-toolbar color="primary">
    <button mat-icon-button aria-label="Show menu" (click)="drawer.toggle()">
      <mat-icon>menu</mat-icon>
    </button>
    <span>Animal Sharing</span>
    <ts-user-bar class="w-100 d-flex flex-row justify-content-end align-items-center"></ts-user-bar>
  </mat-toolbar>
  <mat-drawer-container class="view-container" autosize>
    <mat-drawer #drawer class="sidenav" mode="side" opened>
      <mat-nav-list>
        <a
          *ngFor="let page of pages"
          mat-list-item
          [routerLink]="page.link"
          routerLinkActive="active-page-link"
        >
          {{ page.name }}
        </a>
      </mat-nav-list>
    </mat-drawer>
    <div class="page-container">
      <div class="container my-5">
        <router-outlet></router-outlet>
      </div>
    </div>
  </mat-drawer-container>
\end{lstlisting}

app.component.ts
\lstset{style=ts}
\begin{lstlisting}
    import { DOCUMENT } from '@angular/common';
    import {
      ChangeDetectionStrategy,
      Component,
      Inject,
      OnDestroy,
      OnInit,
      ViewEncapsulation,
    } from '@angular/core';
    import {User, UserService} from '@app/core';
    import {UserStore} from '@app/store/user.store';
    import { Subject } from 'rxjs';
    import {take, takeUntil} from 'rxjs/operators';
    import { ProductGroupService } from './core/api/product-group.service';
    import { ProductGroupStore } from './store/product-group.store';
    import jwt_decode from 'jwt-decode';
    import { CityStore } from './store/city.store';
    import { CityService } from './core/api/city.service';
    
    @Component({
      selector: 'ts-app',
      templateUrl: './app.component.html',
      styleUrls: ['./app.component.scss'],
      encapsulation: ViewEncapsulation.None,
      changeDetection: ChangeDetectionStrategy.Default,
    })
    export class AppComponent implements OnInit, OnDestroy {
      private destroy$ = new Subject<void>();
      pages = [
        {
          link: 'product',
          name: 'Объявления',
        },
        {
          link: 'search',
          name: 'Поиск',
        },
      ];
    
      constructor(
        @Inject(DOCUMENT) private document: Document,
        private readonly productGroupStore: ProductGroupStore,
        private readonly cityStore: CityStore,
        private readonly cityService: CityService,
        private readonly productGroupService: ProductGroupService,
        private readonly userStore: UserStore,
        private readonly userService: UserService,
      ) {}
    
      ngOnInit(): void {
        this.setTheme();
        this.productGroupService
          .getAll()
          .pipe(takeUntil(this.destroy$))
          .subscribe((items) => this.productGroupStore.items$.next(items));
    
        this.cityService.getAll()
          .pipe(takeUntil(this.destroy$))
          .subscribe(
            (cities) => this.cityStore.items$.next(cities),
            (error) => console.error('Error fetching cities', error)
          );
    
        const userToken = localStorage.getItem('token');
        if (userToken) {
          const user = jwt_decode(userToken) as User;
          this.userService.getById(+user.id).pipe(
            take(1),
          ).subscribe(u => {
            this.userStore.user$.next(u);
          });
        }
      }
    
      private setTheme(): void {
        this.document.body.classList.add('default-theme');
      }
    
      ngOnDestroy(): void {
        this.destroy$.next();
        this.destroy$.complete();
      }
    }    
\end{lstlisting}

city.service.ts
\lstset{style=ts}
\begin{lstlisting}
    import { HttpClient } from '@angular/common/http';
    import { Injectable } from '@angular/core';
    import { Observable } from 'rxjs';
    import { City } from '@core/model/City';
    
    @Injectable({
      providedIn: 'root',
    })
    export class CityService {
      private url = '/api/City';
    
      constructor(private http: HttpClient) {}
    
      getAll(): Observable<City[]> {
        return this.http.get<City[]>(this.url);
      }
    }
\end{lstlisting}

product.service.ts
\lstset{style=ts}
\begin{lstlisting}
    import { HttpClient, HttpParams } from '@angular/common/http';
    import { Injectable } from '@angular/core';
    import { Observable } from 'rxjs';
    import { Product, ProductFilter } from '../model/product';
    
    @Injectable({
      providedIn: 'root',
    })
    export class ProductService {
      private url = '/api/Product';
    
      constructor(private http: HttpClient) { }
    
      getAll(): Observable<Product[]> {
        return <Observable<Product[]>> this.http.get(`${this.url}`);
      }
    
      getById(id: number): Observable<Product> {
        return <Observable<Product>> this.http.get(`${this.url}/${id}`);
      }
    
      add(product: Product): Observable<unknown> {
        return <Observable<Product[]>> this.http.post(`${this.url}`, product);
      }
    
      update(product: Product): Observable<unknown> {
        return <Observable<Product[]>> this.http.put(`${this.url}/${product.id}`, product);
      }
    
      delete(id: number): Observable<unknown> {
        return <Observable<Product[]>> this.http.delete(`${this.url}/${id}`);
      }
    
      getFilteredProducts(filter: ProductFilter): Observable<Product[]> {
        let params = new HttpParams();
    
        // Append each filter that's defined to the HttpParams object
        if (filter.cityId) {
          params = params.append('cityId', filter.cityId.toString());
        }
        if (filter.productGroupId) {
          params = params.append('productGroupId', filter.productGroupId.toString());
        }
        if (filter.publishedAtFrom) {
          params = params.append('publishedAtFrom', filter.publishedAtFrom.toISOString());
        }
        if (filter.expiredAtTo) {
          params = params.append('expiredAtTo', filter.expiredAtTo.toISOString());
        }
    
        return this.http.get<Product[]>(`${this.url}/filter`, { params });
      }
    }
\end{lstlisting}

product-group.service.ts
\lstset{style=ts}
\begin{lstlisting}
    import { HttpClient } from '@angular/common/http';
    import { Injectable } from '@angular/core';
    import { Observable } from 'rxjs';
    import { ProductGroup } from '../model/product-group';
    
    @Injectable({
      providedIn: 'root',
    })
    export class ProductGroupService {
      private url = '/api/ProductGroup';
    
      constructor(private http: HttpClient) { }
    
      getAll(): Observable<ProductGroup[]> {
        return <Observable<ProductGroup[]>> this.http.get(this.url);
      }
    
      updateAll(productGroups: ProductGroup[]): Observable<ProductGroup[]> {
        return <Observable<ProductGroup[]>> this.http.post(this.url, productGroups);
      }
    }
\end{lstlisting}

user.service.ts
\lstset{style=ts}
\begin{lstlisting}
    import {HttpClient} from '@angular/common/http';
    import { Injectable } from '@angular/core';
    import {User} from '@app/core';
    import {Observable} from 'rxjs';
    
    @Injectable({
      providedIn: 'root'
    })
    export class UserService {
      private url = '/api/User';
    
      constructor(private http: HttpClient) { }
    
      getById(id: number): Observable<User> {
        return <Observable<User>> this.http.get(`${this.url}/${id}`);
      }
    
      login(user: Partial<User>): Observable<unknown> {
        return <Observable<User[]>> this.http.post(`${this.url}/login`, user);
      }
    
      create(user: Partial<User>): Observable<unknown> {
        return <Observable<User[]>> this.http.post(`${this.url}/create`, user);
      }
    
      update(user: Partial<User>): Observable<User> {
        return <Observable<User>> this.http.patch(`${this.url}/${user.id}`, user);
      }
    }    
\end{lstlisting}

user.store.ts
\lstset{style=ts}
\begin{lstlisting}
    import { Injectable } from '@angular/core';
    import { User } from '@app/core';
    import { BehaviorSubject } from 'rxjs';
    
    @Injectable({
      providedIn: 'root',
    })
    export class UserStore {
      public user$ = new BehaviorSubject<User>(null);
    }    
\end{lstlisting}

user.component.ts
\lstset{style=ts}
\begin{lstlisting}
    import { Component, OnInit, Input } from '@angular/core';
    import { ActivatedRoute } from '@angular/router';
    import { takeUntil, withLatestFrom } from 'rxjs/operators';
    import { Subject } from 'rxjs';
    import {UserService} from "@app/core";
    import {UserStore} from "@app/store/user.store";
    
    @Component({
      selector: 'ts-user',
      templateUrl: './user.component.html',
      styleUrls: ['./user.component.scss']
    })
    export class UserComponent implements OnInit {
      @Input() photoUrl: string;
      @Input() firstName: string;
      @Input() lastName: string;
      @Input() address: string;
      @Input() rating: number;
    
      private destroy$: Subject<void> = new Subject<void>();
    
      constructor(
          private route: ActivatedRoute,
          private userService: UserService,
          private userStore: UserStore
      ) { }
    
      ngOnInit(): void {
        this.route.params.pipe(
            withLatestFrom(this.userStore.user$),
            takeUntil(this.destroy$)
        ).subscribe(([params, userFromStore]) => {
          const userId = +params['id'];
    
          this.userService.getById(userId).subscribe(userFromApi => {
            if (userFromApi && userFromStore) {
              if (userFromApi.id === userFromStore.id) {
              } else {
              }
            }
    
            // Assign user data to component properties
            this.photoUrl = userFromApi.photoUrl;
            this.firstName = userFromApi.firstName;
            this.lastName = userFromApi.lastName;
            this.address = userFromApi.address;
            this.rating = userFromApi.rating;
          });
        });
      }
    
      ngOnDestroy() {
        this.destroy$.next();
        this.destroy$.complete();
      }
    }    
\end{lstlisting}

signup-form.component.html
\begin{lstlisting}
    <mat-card class="signupWrapper">
    <h1 class="regTitle">Как вы хотите зарегистрироваться?</h1>
    <button (click)="openUrl('/signup-client')" class="regLink my-3" mat-stroked-button color="primary">
        Регистрация клиента
    </button>
    <div>или</div>
    <button (click)="openUrl('/signup-sitter')" class="regLink my-3" mat-stroked-button color="primary">
        Регистрация догситтера
    </button>
</mat-card>
\end{lstlisting}

signup.component.ts
\lstset{style=ts}
\begin{lstlisting}
    import { Component, OnInit } from '@angular/core';
    import {Router} from "@angular/router";
    
    @Component({
      selector: 'ts-signup',
      templateUrl: './signup.component.html',
      styleUrls: ['./signup.component.scss']
    })
    export class SignupComponent implements OnInit {
    
      constructor(
          private router: Router,
      ) { }
    
      ngOnInit(): void {
      }
    
      openUrl(url: string) {
        this.router.navigate([url]);
      }}
\end{lstlisting}

signup-form.component.html
\begin{lstlisting}
    <div class="formWrapper">
    <form [formGroup]="userForm" (submit)="onSubmit($event)" class="mainForm">
        <mat-form-field appearance="fill" class="w-100">
            <mat-label>Имя</mat-label>
            <input matInput formControlName="firstName" type="text"/>
        </mat-form-field>
        <mat-form-field appearance="fill" class="w-100">
            <mat-label>Фамилия</mat-label>
            <input matInput formControlName="lastName" type="text"/>
        </mat-form-field>
        <mat-form-field appearance="fill" class="w-100">
            <mat-label>Email</mat-label>
            <input matInput formControlName="email" type="email"/>
            <mat-error *ngIf="userForm.controls.email.hasError('email')">Некорректный формат email</mat-error>
        </mat-form-field>
        <mat-form-field appearance="fill" class="w-100">
          <mat-label>Telegram</mat-label>
          <input matInput formControlName="telegram" type="text"/>
        </mat-form-field>
        <mat-form-field appearance="fill" class="w-100">
            <mat-label>Придумайте пароль</mat-label>
            <input matInput formControlName="password" type="password" />
        </mat-form-field>
        <mat-form-field appearance="fill" class="w-100">
            <mat-label>Повторите пароль</mat-label>
            <input matInput formControlName="password1" type="password" />
        </mat-form-field>
        <mat-checkbox color="primary" formControlName="agreedConditionForm1" class="m-3">Нажимая кнопку «Продолжить» я соглашаюсь с<a href="#">политикой конфиденциальности</a></mat-checkbox>
        <br>
        <button type="submit" mat-raised-button [disabled]="userForm.invalid">Продолжить</button>
    </form>
</div>

<div class="clientCheckboxWrapper">

</div>
\end{lstlisting}


signup-form.component.ts
\lstset{style=ts}
\begin{lstlisting}
    import { Component, Input, OnInit } from '@angular/core';
    import {FormBuilder, Validators} from "@angular/forms";
    import { User, UserService } from "@app/core";
    import {take, tap} from "rxjs/operators";
    import { UserStore } from '@app/store/user.store';
    import { BehaviorSubject } from 'rxjs';
    import { Router } from '@angular/router';
    
    export enum UserRole {
      SITTER = 1,
      OWNER = 2,
    }
    
    @Component({
      selector: 'ts-signup-form',
      templateUrl: './signup-form.component.html',
      styleUrls: ['./signup-form.component.scss']
    })
    export class SignupFormComponent implements OnInit {
      @Input() isSitter = false;
    
      userForm = this.fb.group({
        firstName: ['', Validators.required],
        lastName: ['', Validators.required],
        email: ['', [Validators.required,
          Validators.email]],
        password: ['', [Validators.required,
          Validators.minLength(6),
          Validators.maxLength(24)]],
        password1: ['', [Validators.required,
          Validators.minLength(6),
          Validators.maxLength(24)]],
        agreedConditionForm1: ['', Validators.required],
        telegram: '',
      });
      error: string;
      user$: BehaviorSubject<User> = this.userStore.user$;
    
      constructor(
          private fb: FormBuilder,
          private userService: UserService,
          protected userStore: UserStore,
          private router: Router,
      ) { }
    
      ngOnInit(): void {
      }
    
      onSubmit(_: any) {
        this.userService.create({
          firstName: this.userForm.value.firstName,
          lastName: this.userForm.value.lastName,
          email: this.userForm.value.email,
          telegram: this.userForm.value.telegram,
          password: this.userForm.value.password,
          userRoleId: `${this.isSitter ? UserRole.SITTER : UserRole.OWNER}`,
        }).pipe(
            take(1),
            tap({
              next: (result: User) => {
                this.user$.next(result);
                localStorage.setItem('token', result.accessToken);
                this.router.navigate(['/search']);
              },
              error: err => {
                this.error = err.error.detail;
              }
            }),
        ).subscribe();
      }
    }    
\end{lstlisting}

search.component.html
\begin{lstlisting}
    <h1>Надежные догситтеры и выгульщики собак в вашем районе</h1>
    <p> Позаботимся, пока вы в отъезде  Погуляем, пока вы на работе</p>
    
    <form [formGroup]="searchForm" (ngSubmit)="onSubmit()">
        <mat-form-field class="mx-3">
            <mat-label>Выберите город</mat-label>
            <mat-select formControlName="city">
                <mat-option *ngFor="let city of cityList" [value]="city.id">{{ city.name }}</mat-option>
            </mat-select>
        </mat-form-field>
        <mat-form-field class="mx-3">
            <input matInput [matDatepicker]="publishedDatePicker" placeholder="Начальная дата" formControlName="publishedAtFrom">
            <mat-datepicker-toggle matSuffix [for]="publishedDatePicker"></mat-datepicker-toggle>
            <mat-datepicker #publishedDatePicker></mat-datepicker>
        </mat-form-field>
        <mat-form-field class="mx-3">
            <input matInput [matDatepicker]="expiredDatePicker" placeholder="Конечная дата" formControlName="expiredAtTo">
            <mat-datepicker-toggle matSuffix [for]="expiredDatePicker"></mat-datepicker-toggle>
            <mat-datepicker #expiredDatePicker></mat-datepicker>
        </mat-form-field>
        <mat-form-field class="mx-3">
            <mat-label>Питомец</mat-label>
            <mat-select formControlName="productGroup">
                <mat-option *ngFor="let pg of productGroups" [value]="pg.id">{{ pg.name }}</mat-option>
            </mat-select>
        </mat-form-field>
        <br>
        <button class="mx-3" mat-raised-button type="submit" [disabled]="!searchForm.valid">Поиск</button>
    </form>    
\end{lstlisting}


search.component.ts
\lstset{style=ts}
\begin{lstlisting}
    import { Component, OnDestroy, OnInit } from '@angular/core';
    import { FormGroup, FormControl } from '@angular/forms';
    import { City } from "@core/model/City";
    import { Router } from "@angular/router";
    import { ProductGroupStore } from '@app/store/product-group.store';
    import { Subject } from 'rxjs';
    import { takeUntil } from 'rxjs/operators';
    import { ProductGroup } from '@app/core';
    import { CityStore } from '@app/store/city.store';
    
    @Component({
      selector: 'ts-search',
      templateUrl: './search.component.html',
      styleUrls: ['./search.component.scss'],
    })
    export class SearchComponent implements OnInit, OnDestroy {
      private destroy$ = new Subject<void>();
      searchForm: FormGroup;
      cityList: City[] = [];
      productGroups: ProductGroup[] = [];
    
      constructor(
        private router: Router,
        private readonly productGroupStore: ProductGroupStore,
        private readonly cityStore: CityStore,
      ) {
        this.searchForm = new FormGroup({
          city: new FormControl(''),
          publishedAtFrom: new FormControl(''),
          expiredAtTo: new FormControl(''),
          productGroup: new FormControl(''),
        });
      }
    
      ngOnInit(): void {
        this.productGroupStore.items$
          .pipe(takeUntil(this.destroy$))
          .subscribe((items) => (this.productGroups = items));
    
        this.cityStore.items$
          .pipe(takeUntil(this.destroy$))
          .subscribe((items) => (this.cityList = items));
      }
    
      onSubmit() {
        // Get form values and format dates
        const { city, publishedAtFrom, expiredAtTo, productGroup } = this.searchForm.value;
        const formattedPublishedAtFrom = publishedAtFrom ? new Date(publishedAtFrom).toISOString().split('T')[0] : null;
        const formattedExpiredAtTo = expiredAtTo ? new Date(expiredAtTo).toISOString().split('T')[0] : null;
    
        // Navigate with query parameters
        this.router.navigate(['/search-results'], {
          queryParams: {
            cityId: city, // assuming the city field contains the city ID
            productGroupId: productGroup,
            publishedAtFrom: formattedPublishedAtFrom,
            expiredAtTo: formattedExpiredAtTo,
          }
        });
      }
    
      ngOnDestroy(): void {
        this.destroy$.next();
        this.destroy$.complete();
      }
    }    
\end{lstlisting}

search-results.component.html
\begin{lstlisting}
    <h1 class="text-center title">
    Результаты поиска
  </h1>
  
  <div class="row main-row">
    <div *ngFor="let product of products; trackBy: trackById" class="ts-card-wrapper">
      <ts-card
        class="ts-card"
        *ngIf="product"
        [isEditable]="(user$ | async)?.id === product.createdByUserId"
        [title]="product.name"
        [author]="product.createdByUser"
        [imageUrl]="product.imageUrl"
        [description]="product.description"
        [link]="'/product/' + product.id"
        [publishedAt]="product.publishedAt"
        [expiredAt]="product.expiredAt"
        [price]="product.price"
        [cityId]="product.cityId"
        (onClickDelete)="deleteProduct(product.id)"
      ></ts-card>
    </div>
  </div>  
\end{lstlisting}


search-results.component.ts
\lstset{style=ts}
\begin{lstlisting}
    import { Component, OnDestroy, OnInit } from '@angular/core';
    import { ActivatedRoute } from '@angular/router';
    import { Product, ProductService, User, ProductFilter } from '@app/core';
    import { UserStore } from '@app/store/user.store';
    import { BehaviorSubject, Subject } from 'rxjs';
    import { takeUntil } from 'rxjs/operators';
    
    @Component({
      selector: 'ts-search-results',
      templateUrl: './search-results.component.html',
      styleUrls: ['./search-results.component.scss']
    })
    export class SearchResultsComponent implements OnInit, OnDestroy {
      private destroy$ = new Subject<void>();
      products: Product[] = [];
      user$: BehaviorSubject<User> = this.userStore.user$;
      filter: ProductFilter;
    
      constructor(
        private readonly productService: ProductService,
        private readonly userStore: UserStore,
        private route: ActivatedRoute,
      ) {}
    
      ngOnInit() {
        this.route.queryParams.subscribe(params => {
          const filter: ProductFilter = {};
    
          if (params['cityId']) {
            filter.cityId = +params['cityId'];
          }
          if (params['productGroupId']) {
            filter.productGroupId = +params['productGroupId'];
          }
          if (params['publishedAtFrom']) {
            filter.publishedAtFrom = new Date(params['publishedAtFrom']);
          }
          if (params['expiredAtTo']) {
            filter.expiredAtTo = new Date(params['expiredAtTo']);
          }
    
          this.filter = filter;
    
          this.updateProducts();
        });
      }
    
      private updateProducts(): void {
        this.productService
          .getFilteredProducts(this.filter)
          .pipe(takeUntil(this.destroy$))
          .subscribe((products) => {
            this.products = products;
          });
      }
    
      deleteProduct(id: number): void {
        this.productService
          .delete(id)
          .pipe(takeUntil(this.destroy$))
          .subscribe(() => this.updateProducts());
      }
    
      trackById = (index: number, item: Product) => (item ? item.id : index);
    
      ngOnDestroy() {
        this.destroy$.next();
        this.destroy$.complete();
      }
    }    
\end{lstlisting}

product-routing.module.ts
\lstset{style=ts}
\begin{lstlisting}
    import { NgModule } from '@angular/core';
    import { RouterModule, Routes } from '@angular/router';
    import { ProductCreatePage } from './product-create/product-create.page';
    import { ProductListPage } from './product-list.page';
    import { ProductViewPage } from './product-view/product-view.page';
    
    const routes: Routes = [
      {
        path: '',
        component: ProductListPage,
      },
      {
        path: 'add',
        component: ProductCreatePage,
      },
      {
        path: ':id',
        component: ProductViewPage,
      }
    ];
    
    @NgModule({
      imports: [RouterModule.forChild(routes)],
      exports: [RouterModule],
    })
    export class ProductRoutingModule {
    }    
\end{lstlisting}

product-list.page.html
\begin{lstlisting}
    <h1 class="text-center title">
    Список объявлений
    <a *ngIf="user$ | async"
       mat-raised-button
       class="create-item-button"
       [routerLink]="['/product/add']">ДОБАВИТЬ</a>
  </h1>
  <div class="row main-row">
    <div *ngFor="let product of products; trackBy: trackById"
        class="ts-card-wrapper">
      <ts-card
        class="ts-card"
        [isEditable]="(user$ | async)?.id === product.createdByUserId"
        [title]="product.name"
        [author]="product.createdByUser"
        [imageUrl]="product.imageUrl"
        [description]="product.description"
        [link]="'/product/' + product.id"
        [price]="product.price"
        [publishedAt]="product.publishedAt"
        [expiredAt]="product.expiredAt"
        [cityId]="product.cityId"
        (onClickDelete)="deleteProduct(product.id)"
      ></ts-card>
    </div>
  </div>
\end{lstlisting}


product-list.page.ts
\lstset{style=ts}
\begin{lstlisting}
    import {
        ChangeDetectionStrategy,
        Component,
        OnDestroy,
        OnInit,
        ViewEncapsulation,
      } from '@angular/core';
      import {User} from '@app/core';
      import { ProductService } from '@app/core/api/product.service';
      import { Product } from '@app/core/model/product';
      import {UserStore} from '@app/store/user.store';
      import {BehaviorSubject, Subject} from 'rxjs';
      import { takeUntil } from 'rxjs/operators';
      
      @Component({
        selector: 'ts-product-list',
        templateUrl: './product-list.page.html',
        styleUrls: ['./product-list.page.scss'],
        encapsulation: ViewEncapsulation.None,
        changeDetection: ChangeDetectionStrategy.Default,
      })
      export class ProductListPage implements OnInit, OnDestroy {
        private destroy$ = new Subject<void>();
        products: Product[] = [];
        user$: BehaviorSubject<User> = this.userStore.user$;
      
        constructor(
          private readonly productService: ProductService,
          private readonly userStore: UserStore,
        ) {}
      
        ngOnInit() {
          this.updateProducts();
        }
      
        private updateProducts(): void {
          this.productService
            .getAll()
            .pipe(takeUntil(this.destroy$))
            .subscribe((products) => {
              this.products = products;
            });
        }
      
        deleteProduct(id: number): void {
          this.productService
            .delete(id)
            .pipe(takeUntil(this.destroy$))
            .subscribe(() => this.updateProducts());
        }
      
        trackById = (index: number, item: Product) => (item ? item.id : index);
      
        ngOnDestroy() {
          this.destroy$.next();
          this.destroy$.complete();
        }
      }      
\end{lstlisting}

product-view.page.html
\begin{lstlisting}
    <ng-container *ngIf="product && productForm">
    <h1 class="text-center">
      {{ product.name }}
    </h1>
  
    <img mat-card-image class="w-100 mb-3"
    [src]="product.imageUrl"
    alt="Photo of an item in list">
  
    <form [formGroup]="productForm">
      <mat-form-field appearance="fill" class="w-100">
        <mat-label>Название</mat-label>
        <input matInput formControlName="name" />
      </mat-form-field>
    
      <mat-form-field appearance="fill" class="w-100">
        <mat-label>Описание</mat-label>
        <textarea matInput formControlName="description"></textarea>
      </mat-form-field>
    
      <mat-form-field appearance="fill" class="w-100">
        <mat-label>Категория</mat-label>
        <mat-select formControlName="productGroupId">
          <mat-option
            *ngFor="let productGroup of productGroups"
            [value]="productGroup.id"
          >
            {{ productGroup.name }}
          </mat-option>
        </mat-select>
      </mat-form-field>
  
      <mat-form-field appearance="fill" class="w-100">
        <input matInput [matDatepicker]="publishedDatePicker" placeholder="Начальная дата" formControlName="publishedAt">
        <mat-datepicker-toggle matSuffix [for]="publishedDatePicker"></mat-datepicker-toggle>
        <mat-datepicker #publishedDatePicker></mat-datepicker>
      </mat-form-field>
      <mat-form-field appearance="fill" class="w-100">
          <input matInput [matDatepicker]="expiredDatePicker" placeholder="Конечная дата" formControlName="expiredAt">
          <mat-datepicker-toggle matSuffix [for]="expiredDatePicker"></mat-datepicker-toggle>
          <mat-datepicker #expiredDatePicker></mat-datepicker>
      </mat-form-field>
  
      <mat-form-field appearance="fill" class="w-100">
        <mat-label>Выберите город</mat-label>
        <mat-select formControlName="cityId">
            <mat-option *ngFor="let city of cityList" [value]="city.id">{{ city.name }}</mat-option>
        </mat-select>
      </mat-form-field>
    
      <mat-form-field appearance="fill" class="w-100">
        <mat-label>Цена</mat-label>
        <input matInput formControlName="price" />
      </mat-form-field>
    
    </form>
    
    <mat-form-field appearance="fill" class="w-100">
      <div>
        <mat-toolbar>
          <input matInput [(ngModel)]="fileAttr" readonly name="name" />
          <button mat-flat-button color="primary">Выбрать</button>
        </mat-toolbar>
    
        <input
          type="file"
          #fileInput
          id="uploadFile"
          (change)="uploadFileEvt($event)"
          name="uploadFile"
          accept="image/*"
        />
      </div>
    </mat-form-field>
    
    <button mat-raised-button (click)="refresh()" color="warn" class="mr-3">ОТМЕНИТЬ</button>
    <button mat-raised-button (click)="saveProduct()">СОХРАНИТЬ</button>
  </ng-container>  
\end{lstlisting}


product-view.page.ts
\lstset{style=ts}
\begin{lstlisting}
    import { Component, ElementRef, OnDestroy, OnInit, ViewChild } from '@angular/core';
    import { FormBuilder, FormGroup, Validators } from '@angular/forms';
    import { ActivatedRoute } from '@angular/router';
    import { ProductService } from '@app/core';
    import { Product } from '@app/core/model/product';
    import { ProductGroupStore } from '@app/store/product-group.store';
    import { Subject } from 'rxjs';
    import { takeUntil } from 'rxjs/operators';
    import { defaultFileAttr } from '../product-create/product-create.page';
    import { City } from '@app/core/model/City';
    import { CityStore } from '@app/store/city.store';
    
    @Component({
      selector: 'ts-product-view',
      templateUrl: './product-view.page.html',
      styleUrls: ['./product-view.page.scss'],
    })
    export class ProductViewPage implements OnInit, OnDestroy {
      private destroy$ = new Subject<void>();
      product: Product;
      productForm: FormGroup;
      productGroups = [];
      private imgBase64Path: string;
      @ViewChild('fileInput') fileInput: ElementRef;
      fileAttr = defaultFileAttr;
      cityList: City[] = [];
    
      constructor(
        private readonly productService: ProductService,
        private readonly route: ActivatedRoute,
        private readonly fb: FormBuilder,
        private readonly productGroupStore: ProductGroupStore,
        private readonly cityStore: CityStore,
      ) {}
    
      ngOnInit(): void {
        this.refresh();
        this.productGroupStore.items$
          .pipe(takeUntil(this.destroy$))
          .subscribe((items) => (this.productGroups = items));
        this.cityStore.items$
          .pipe(takeUntil(this.destroy$))
          .subscribe((items) => (this.cityList = items));
      }
    
      refresh(): void {
        this.productForm = this.fb.group({
          name: ['', Validators.required],
          description: ['', Validators.required],
          productGroupId: [null, Validators.required],
          price: [null, Validators.required],
          publishedAt: [null, Validators.required],
          expiredAt: [null, Validators.required],
          cityId: [null, Validators.required],
        });
    
        const productId = this.route.snapshot.paramMap.get('id');
        this.productService
          .getById(+productId)
          .pipe(takeUntil(this.destroy$))
          .subscribe((product) => {
            this.product = product;
            this.productForm.patchValue(product);
            this.imgBase64Path = null;
            this.fileAttr = defaultFileAttr;
          });
      }
    
      saveProduct(): void {
        if (!this.productForm.valid) {
          return;
        }
        this.productService
          .update({
            ...this.product,
            ...this.productForm.value,
            imageUrl: this.imgBase64Path || this.product.imageUrl,
          })
          .subscribe(() => this.refresh());
      }
    
      uploadFileEvt(imgFile: any) {
        if (imgFile.target.files && imgFile.target.files[0]) {
          this.fileAttr = '';
          Array.from(imgFile.target.files).forEach((file: File) => {
            this.fileAttr += file.name;
          });
    
          const reader = new FileReader();
          reader.onload = (e: any) => {
            const image = new Image();
            image.src = e.target.result;
            image.onload = () => {
              this.imgBase64Path = e.target.result;
            };
          };
          reader.readAsDataURL(imgFile.target.files[0]);
    
          this.fileInput.nativeElement.value = '';
        } else {
          this.imgBase64Path = null;
          this.fileAttr = defaultFileAttr;
        }
      }
    
      ngOnDestroy() {
        this.destroy$.next();
        this.destroy$.complete();
      }
    }    
\end{lstlisting}

product-create.page.html
\begin{lstlisting}
    <h1 class="text-center">Создание объявления</h1>
    <form [formGroup]="productForm">
      <mat-form-field appearance="fill" class="w-100">
        <mat-label>Название</mat-label>
        <input matInput formControlName="name" />
      </mat-form-field>
    
      <mat-form-field appearance="fill" class="w-100">
        <mat-label>Описание</mat-label>
        <textarea matInput formControlName="description"></textarea>
      </mat-form-field>
    
      <mat-form-field appearance="fill" class="w-100">
        <mat-label>Категория</mat-label>
        <mat-select formControlName="productGroupId">
          <mat-option
            *ngFor="let productGroup of productGroups"
            [value]="productGroup.id"
          >
            {{ productGroup.name }}
          </mat-option>
        </mat-select>
      </mat-form-field>
    
        <mat-form-field appearance="fill" class="w-100">
          <input matInput [matDatepicker]="publishedDatePicker" placeholder="Начальная дата" formControlName="publishedAt">
          <mat-datepicker-toggle matSuffix [for]="publishedDatePicker"></mat-datepicker-toggle>
          <mat-datepicker #publishedDatePicker></mat-datepicker>
      </mat-form-field>
      <mat-form-field appearance="fill" class="w-100">
          <input matInput [matDatepicker]="expiredDatePicker" placeholder="Конечная дата" formControlName="expiredAt">
          <mat-datepicker-toggle matSuffix [for]="expiredDatePicker"></mat-datepicker-toggle>
          <mat-datepicker #expiredDatePicker></mat-datepicker>
      </mat-form-field>
    
      <mat-form-field appearance="fill" class="w-100">
        <mat-label>Выберите город</mat-label>
        <mat-select formControlName="cityId">
            <mat-option *ngFor="let city of cityList" [value]="city.id">{{ city.name }}</mat-option>
        </mat-select>
      </mat-form-field>
    
      <mat-form-field appearance="fill" class="w-100">
        <mat-label>Цена</mat-label>
        <input matInput formControlName="price" />
      </mat-form-field>
    </form>
    
    <mat-form-field appearance="fill" class="w-100">
      <div>
        <mat-toolbar>
          <input matInput [(ngModel)]="fileAttr" readonly name="name" />
          <button mat-flat-button color="primary">Выбрать</button>
        </mat-toolbar>
    
        <input
          type="file"
          #fileInput
          id="uploadFile"
          (change)="uploadFileEvt($event)"
          name="uploadFile"
          accept="image/*"
        />
      </div>
    </mat-form-field>
    
    <button mat-raised-button (click)="createProduct()">СОЗДАТЬ</button>    
\end{lstlisting}


product-create.page.ts
\lstset{style=ts}
\begin{lstlisting}
    import { Location } from '@angular/common';
    import { Component, ElementRef, OnDestroy, OnInit, ViewChild } from '@angular/core';
    import { FormBuilder, FormGroup, Validators } from '@angular/forms';
    import { ProductService } from '@app/core';
    import { City } from '@app/core/model/City';
    import { CityStore } from '@app/store/city.store';
    import { ProductGroupStore } from '@app/store/product-group.store';
    import { Subject } from 'rxjs';
    import { takeUntil } from 'rxjs/operators';
    
    export const defaultFileAttr = 'Изображение';
    
    @Component({
      selector: 'ts-product-create',
      templateUrl: './product-create.page.html',
      styleUrls: ['./product-create.page.scss'],
    })
    export class ProductCreatePage implements OnInit, OnDestroy {
      private destroy$ = new Subject<void>();
      productForm: FormGroup;
      productGroups = [];
      private imgBase64Path: string;
      @ViewChild('fileInput') fileInput: ElementRef;
      fileAttr = defaultFileAttr;
      cityList: City[] = [];
    
      constructor(
        private readonly fb: FormBuilder,
        private readonly productService: ProductService,
        private location: Location,
        private readonly productGroupStore: ProductGroupStore,
        private readonly cityStore: CityStore,
      ) {}
    
      ngOnInit(): void {
        this.productGroupStore.items$
          .pipe(takeUntil(this.destroy$))
          .subscribe((items) => (this.productGroups = items));
        this.productForm = this.fb.group({
          name: ['', Validators.required],
          description: ['', Validators.required],
          productGroupId: [null, Validators.required],
          price: [null, Validators.required],
          publishedAt: [null, Validators.required],
          expiredAt: [null, Validators.required],
          cityId: [null, Validators.required],
        });
        this.cityStore.items$
          .pipe(takeUntil(this.destroy$))
          .subscribe((items) => (this.cityList = items));
      }
    
      createProduct() {
        if (!this.productForm.valid || !this.imgBase64Path) {
          return;
        }
        this.productService
          .add({
            ...this.productForm.value,
            imageUrl: this.imgBase64Path,
          })
          .subscribe(() => this.location.back());
      }
    
      uploadFileEvt(imgFile: any) {
        if (imgFile.target.files && imgFile.target.files[0]) {
          this.fileAttr = '';
          Array.from(imgFile.target.files).forEach((file: File) => {
            this.fileAttr += file.name;
          });
    
          const reader = new FileReader();
          reader.onload = (e: any) => {
            const image = new Image();
            image.src = e.target.result;
            image.onload = () => {
              this.imgBase64Path = e.target.result;
            };
          };
          reader.readAsDataURL(imgFile.target.files[0]);
    
          this.fileInput.nativeElement.value = '';
        } else {
          this.imgBase64Path = null;
          this.fileAttr = defaultFileAttr;
        }
      }
    
      ngOnDestroy(): void {
        this.destroy$.next();
        this.destroy$.complete();
      }
    }    
\end{lstlisting}

login.component.html
\begin{lstlisting}
    <h1 mat-dialog-title>Пожалуйста авторизуйтесь</h1>
    <div mat-dialog-content>
        <form [formGroup]="userForm">
            <mat-form-field appearance="fill" class="w-100">
                <mat-label>Почта</mat-label>
                <input matInput formControlName="email" type="email" />
            </mat-form-field>
    
            <mat-form-field appearance="fill" class="w-100">
                <mat-label>Пароль</mat-label>
                <input matInput formControlName="password" type="password" />
            </mat-form-field>
    
            <mat-error *ngIf="error">{{error}}</mat-error>
        </form>
    </div>
    <div mat-dialog-actions>
        <button mat-button (click)="onNoClick()">Назад</button>
        <button mat-button (click)="onOkClick()" cdkFocusInitial>Далее</button>
    </div>    
\end{lstlisting}


login.component.ts
\lstset{style=ts}
\begin{lstlisting}
    import { Component } from '@angular/core';
    import { FormBuilder, Validators } from '@angular/forms';
    import { User, UserService } from '@app/core';
    import { take, tap } from 'rxjs/operators';
    import { BehaviorSubject } from 'rxjs';
    import { UserStore } from '@app/store/user.store';
    import { Router } from '@angular/router';
    
    @Component({
      selector: 'ts-login',
      templateUrl: './login.component.html',
      styleUrls: ['./login.component.scss'],
    })
    export class LoginComponent {
      userForm = this.fb.group({
        email: ['', Validators.required],
        password: ['', Validators.required],
      });
      error: string;
      user$: BehaviorSubject<User> = this.userStore.user$;
    
      constructor(private fb: FormBuilder,
                  private userService: UserService,
                  private readonly userStore: UserStore,
                  private router: Router,
                  ) {}
    
      onOkClick(): void {
        this.error = null;
        this.userForm.updateValueAndValidity();
        this.userForm.markAllAsTouched();
    
        if (this.userForm.invalid) {
          return;
        }
    
        this.userService.login(this.userForm.value)
            .pipe(
                take(1),
                tap({
                  next: (result: User) => {
                    this.user$.next(result);
                    localStorage.setItem('token', result.accessToken);
                    this.router.navigate(['/search']);
                  },
                  error: err => {
                    this.error = err.error.detail;
                  },
                })
            )
            .subscribe();
      }
    }    
\end{lstlisting}

card.component.html
\begin{lstlisting}
    <mat-card class="card">
    <mat-card-header class="card__header">
      <div mat-card-avatar class="card__header-avatar">
        <img *ngIf="author.imageUrl"
              mat-card-image
              [src]="imageUrl"
              alt="Photo of an item in list">
        <mat-icon *ngIf="!author.imageUrl"
                  mat-card-avatar
        >supervised_user_circle</mat-icon>
      </div>
  
      <mat-card-title>
        {{ title }}
      </mat-card-title>
  
      <mat-card-subtitle>
        {{ author.firstName + ' ' + author.lastName }}
      </mat-card-subtitle>
  
      <a *ngIf="isEditable"
         [routerLink]="[link]"
         mat-icon-button
         class="edit-button">
        <mat-icon>edit</mat-icon>
      </a>
  
      <button *ngIf="isEditable"
              (click)="onClickDelete.emit()"
              mat-icon-button
              class="delete-button"
              color="warn">
        <mat-icon>delete</mat-icon>
      </button>
  
    </mat-card-header>
    <img mat-card-image
         [src]="imageUrl"
         alt="Photo of an item in list">
    <mat-card-content>
      <p>
        {{formatDateRange(publishedAt, expiredAt)}}
      </p>
  
      <p class="mb-0" style="font-size: 1.8em;">
        {{getCityNameById(cityId)}}
      </p>
      
      <mat-expansion-panel
        hideToggle
        class="more-info"
        (opened)="panelOpenState = true"
        (closed)="panelOpenState = false"
      >
        <mat-expansion-panel-header class="more-info__header p-0">
          <mat-panel-title class="font-weight-bold">
            {{ panelOpenState ? 'свернуть' : 'подробнее...' }}
          </mat-panel-title>
        </mat-expansion-panel-header>
        <div class="more-info__content">
          <p>
            {{ description }}
          </p>
  
          <p>
            <span class="font-weight-bold">Телеграм:</span>
            {{ author.telegram }}
          </p>
          <p>
            <span class="font-weight-bold">Почта:</span>
            {{ author.email }}
          </p>
        </div>
      </mat-expansion-panel>
      <div class="d-flex flex-direction-rowd-flex flex-direction-row justify-content-between w-100 align-items-center">
        <h2 class="font-weight-bold m-0">{{ price }} ₽</h2>
        <!-- <span>{{ weight }} кг.</span> -->
      </div>
    </mat-card-content>
    <mat-card-actions class="text-center">
    </mat-card-actions>
  </mat-card>  
\end{lstlisting}


card.component.ts
\lstset{style=ts}
\begin{lstlisting}
    import {
        ChangeDetectionStrategy,
        Component,
        EventEmitter,
        Input,
        Output,
        ViewEncapsulation,
      } from '@angular/core';
      import {User} from '@app/core';
      import { CityStore } from '@app/store/city.store';
      import { formatDateRange } from '@app/util/date';
      
      @Component({
        selector: 'ts-card',
        templateUrl: './card.component.html',
        styleUrls: ['./card.component.scss'],
        encapsulation: ViewEncapsulation.None,
        changeDetection: ChangeDetectionStrategy.OnPush,
      })
      export class CardComponent {
        @Input() isEditable = false;
        @Input() imageUrl = '';
        @Input() author: User;
        @Input() title: string;
        @Input() description: string;
        @Input() link: string;
        @Input() price: number;
        @Input() publishedAt: string;
        @Input() expiredAt: string;
        @Input() cityId: number;
        @Output() onClickDelete = new EventEmitter();
      
        panelOpenState = false;
        formatDateRange = formatDateRange;
      
        constructor(
          readonly cityStore: CityStore, 
        ) {
        }
      
        getCityNameById(id: number): string {
          const city = this.cityStore.items$.value.find((city) => city.id === id);
          return city ? city.name : '';
        }
      }      
\end{lstlisting}


\ifВКР{
\newpage
\addcontentsline{toc}{section}{На отдельных листах (CD-RW в прикрепленном конверте)}
\begin{center}
\textbf{Место для диска}
\end{center}
}\fi
