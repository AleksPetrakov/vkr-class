\section{Рабочий проект}
\subsection{Описание REST API приложения}

Можно выделить следующий перечень HTTP-методов, использованных при разработке веб-приложения и доступных для использования клиентской частью приложения. Описание этих методов предоставлены в виде аблиц с уменьшенным межстрочным интервалом:
\begin{itemize}
    \item описания методов для работы с городами (таблица \ref{city:table});
    \item описание методов для работы с карточками товаров и услуг (таблица \ref{product:table});
    \item описание методов для работы с типами товаров и услуг (таблица \ref{type:table});
    \item описание методов для работы с данными пользователя (таблица \ref{data:table}).
\end{itemize}

\renewcommand{\arraystretch}{0.8} % уменьшение расстояний до сетки таблицы

\begin{xltabular}{\textwidth}{|X|p{2.5cm}|>{\setlength{\baselineskip}{0.7\baselineskip}}p{4.85cm}|>{\setlength{\baselineskip}{0.7\baselineskip}}p{4.85cm}|}
\caption{Описания методов для работы с городами\label{city:table}}\\
\hline \centrow \setlength{\baselineskip}{0.7\baselineskip} HTTP-метод & \centrow \setlength{\baselineskip}{0.7\baselineskip} Описание & \centrow Входные параметры & \centrow Пример JSON ответа \\
\hline \centrow 1 & \centrow 2 & \centrow 3 & \centrow 4\\ \hline
\endfirsthead
\caption*{Продолжение таблицы \ref{city:table}}\\
\hline \centrow 1 & \centrow 2 & \centrow 3 & \centrow 4\\ \hline
\finishhead
GET /api/City & Список всех городов City в БД & Нет & [
    \{
      "id": 0,
      "name": "string",
      "description": "string"
    \}
  ]\\
\end{xltabular}

\begin{xltabular}{\textwidth}{|X|p{2.5cm}|>{\setlength{\baselineskip}{0.7\baselineskip}}p{4.85cm}|>{\setlength{\baselineskip}{0.7\baselineskip}}p{4.85cm}|}
    \caption{Описание методов для работы с карточками товаров и услуг\label{product:table}}\\
    \hline \centrow \setlength{\baselineskip}{0.7\baselineskip} HTTP-метод & \centrow \setlength{\baselineskip}{0.7\baselineskip} Описание & \centrow Входные параметры & \centrow Пример JSON ответа \\
    \hline \centrow 1 & \centrow 2 & \centrow 3 & \centrow 4\\ \hline
    \endfirsthead
    \caption*{Продолжение таблицы \ref{product:table}}\\
    \hline \centrow 1 & \centrow 2 & \centrow 3 & \centrow 4\\ \hline
    \finishhead
    GET /api/Product & Список всех карточек Product & Нет & [
        \{
          "id": 0,
          "name": "string",
          "description": "string",
          "imageUrl": "string",
          "price": 0,
          "createdByUserId": 0,
          "cityId": 0,
          "productGroupId": 0,
          "productStatusId": 0,
          "createdAt": "2024-01-20T14:39:25.508Z",
          "publishedAt": "2024-01-20T14:39:25.508Z",
          "expiredAt": "2024-01-20T14:39:25.508Z",
          "priorityId": 0,
          "priorityStartedAt": "2024-01-20T14:39:25.508Z",
          "priorityExpiredAt": "2024-01-20T14:39:25.508Z",
          "createdByUser": \{
            "phone": "string",
            "email": "string",
            "firstName": "string",
            "lastName": "string",
            "imageUrl": "string",
            "telegram": "string",
            "website": "string",
            "userRoleId": "string",
            "permissions": "string",
            "id": 0
          \}
        \}
      ]\\
\hline POST /api/Product & Создать новый Product & Body: \{
    "id": 0,
    "name": "string",
    "description": "string",
    "imageUrl": "string",
    "price": 0,
    "createdByUserId": 0,
    "cityId": 0,
    "productGroupId": 0,
    "productStatusId": 0,
    "createdAt": "2024-01-20T14:40:49.706Z",
    "publishedAt": "2024-01-20T14:40:49.706Z",
    "expiredAt": "2024-01-20T14:40:49.706Z",
    "priorityId": 0,
    "priorityStartedAt": "2024-01-20T14:40:49.706Z",
    "priorityExpiredAt": "2024-01-20T14:40:49.706Z",
    "createdByUser": \{
      "phone": "string",
      "email": "string",
      "firstName": "string",
      "lastName": "string",
      "imageUrl": "string",
      "telegram": "string",
      "website": "string",
      "userRoleId": "string",
      "permissions": "string",
      "id": 0
    \}
  \} & \{
    "id": 0,
    "name": "string",
    "description": "string",
    "imageUrl": "string",
    "price": 0,
    "createdByUserId": 0,
    "cityId": 0,
    "productGroupId": 0,
    "productStatusId": 0,
    "createdAt": "2024-01-20T14:40:49.721Z",
    "publishedAt": "2024-01-20T14:40:49.721Z",
    "expiredAt": "2024-01-20T14:40:49.721Z",
    "priorityId": 0,
    "priorityStartedAt": "2024-01-20T14:40:49.721Z",
    "priorityExpiredAt": "2024-01-20T14:40:49.721Z"
  \}\\
\hline GET /api/Product/\{id\} & Получить выбранный Product по его id & Нет & \{
    "id": 0,
    "name": "string",
    "description": "string",
    "imageUrl": "string",
    "price": 0,
    "createdByUserId": 0,
    "cityId": 0,
    "productGroupId": 0,
    "productStatusId": 0,
    "createdAt": "2024-01-20T14:44:34.815Z",
    "publishedAt": "2024-01-20T14:44:34.815Z",
    "expiredAt": "2024-01-20T14:44:34.815Z",
    "priorityId": 0,
    "priorityStartedAt": "2024-01-20T14:44:34.815Z",
    "priorityExpiredAt": "2024-01-20T14:44:34.815Z"
  \}\\
\hline PUT /api/Product/\{id\} & Обновить существующий Product & Body: \{
    "id": 0,
    "name": "string",
    "description": "string",
    "imageUrl": "string",
    "price": 0,
    "createdByUserId": 0,
    "cityId": 0,
    "productGroupId": 0,
    "productStatusId": 0,
    "createdAt": "2024-01-20T14:50:16.671Z",
    "publishedAt": "2024-01-20T14:50:16.671Z",
    "expiredAt": "2024-01-20T14:50:16.671Z",
    "priorityId": 0,
    "priorityStartedAt": "2024-01-20T14:50:16.671Z",
    "priorityExpiredAt": "2024-01-20T14:50:16.671Z",
    "createdByUser": \{
      "phone": "string",
      "email": "string",
      "firstName": "string",
      "lastName": "string",
      "imageUrl": "string",
      "telegram": "string",
      "website": "string",
      "userRoleId": "string",
      "permissions": "string",
      "id": 0
    \}
  \} & \{
    "id": 0,
    "name": "string",
    "description": "string",
    "imageUrl": "string",
    "price": 0,
    "createdByUserId": 0,
    "cityId": 0,
    "productGroupId": 0,
    "productStatusId": 0,
    "createdAt": "2024-01-20T14:50:16.673Z",
    "publishedAt": "2024-01-20T14:50:16.673Z",
    "expiredAt": "2024-01-20T14:50:16.673Z",
    "priorityId": 0,
    "priorityStartedAt": "2024-01-20T14:50:16.673Z",
    "priorityExpiredAt": "2024-01-20T14:50:16.673Z"
  \}\\
\hline DELETE /api/Product/\{id\} & Удалить Product по его id & Нет & Нет\\
\hline GET /api/Product/filter & Получить отфильтрованный список существующих продуктов & Query parameters: cityId: integer, productGroupId: integer, publishedAtFrom: string, expiredAtTo: string. & [
    \{
      "id": 0,
      "name": "string",
      "description": "string",
      "imageUrl": "string",
      "price": 0,
      "createdByUserId": 0,
      "cityId": 0,
      "productGroupId": 0,
      "productStatusId": 0,
      "createdAt": "2024-01-20T14:54:37.620Z",
      "publishedAt": "2024-01-20T14:54:37.620Z",
      "expiredAt": "2024-01-20T14:54:37.620Z",
      "priorityId": 0,
      "priorityStartedAt": "2024-01-20T14:54:37.620Z",
      "priorityExpiredAt": "2024-01-20T14:54:37.620Z"
    \}
  ]\\
% \hline POST /api/Product & Создать новый Product & Body: & Нет \\
\end{xltabular}

\begin{xltabular}{\textwidth}{|X|p{2.5cm}|>{\setlength{\baselineskip}{0.7\baselineskip}}p{4.85cm}|>{\setlength{\baselineskip}{0.7\baselineskip}}p{4.85cm}|}
    \caption{Описание методов для работы с типами товаров и услуг\label{type:table}}\\
    \hline \centrow \setlength{\baselineskip}{0.7\baselineskip} HTTP-метод & \centrow \setlength{\baselineskip}{0.7\baselineskip} Описание & \centrow Входные параметры & \centrow JSON ответ \\
    \hline \centrow 1 & \centrow 2 & \centrow 3 & \centrow 4\\ \hline
    \endfirsthead
    \caption*{Продолжение таблицы \ref{type:table}}\\
    \hline \centrow 1 & \centrow 2 & \centrow 3 & \centrow 4\\ \hline
    \finishhead
    GET /api/ProductGtoup & Список всех типов товаров и услуг ProductGroup в БД & Нет & [
        {
          "id": 0,
          "name": "string",
          "description": "string",
          "type": 0,
          "childrenProductGroupIds": [
            0
          ],
          "parentProductGroupId": 0
        }
      ]\\
      \hline POST /api/ProductGtoup & Создать новый ProductGtoup & Body: [
        \{
          "id": 0,
          "name": "string",
          "description": "string",
          "type": 0,
          "childrenProductGroupIds": [
            0
          ],
          "parentProductGroupId": 0
        \}
      ] & [
        \{
          "id": 0,
          "name": "string",
          "description": "string",
          "type": 0,
          "childrenProductGroupIds": [
            0
          ],
          "parentProductGroupId": 0
        \}
      ] \\
\end{xltabular}

% \hline POST /api/Product & Создать новый Product & Body: & Нет \\

\begin{xltabular}{\textwidth}{|X|p{2.5cm}|>{\setlength{\baselineskip}{0.7\baselineskip}}p{4.85cm}|>{\setlength{\baselineskip}{0.7\baselineskip}}p{4.85cm}|}
    \caption{Описание методов для работы с данными пользователя\label{data:table}}\\
    \hline \centrow \setlength{\baselineskip}{0.7\baselineskip} HTTP-метод & \centrow \setlength{\baselineskip}{0.7\baselineskip} Описание & \centrow Входные параметры & \centrow JSON ответ \\
    \hline \centrow 1 & \centrow 2 & \centrow 3 & \centrow 4\\ \hline
    \endfirsthead
    \caption*{Продолжение таблицы \ref{data:table}}\\
    \hline \centrow 1 & \centrow 2 & \centrow 3 & \centrow 4\\ \hline
    \finishhead
    GET /api/User/{id} & Получить данные существующего пользователя User по его id & Нет & \{
        "phone": "string",
        "email": "string",
        "firstName": "string",
        "lastName": "string",
        "imageUrl": "string",
        "telegram": "string",
        "website": "string",
        "userRoleId": "string",
        "permissions": "string",
        "id": 0
      \}\\
      \hline POST /api/User/create & Создать нового пользователя User & Body: \{
        "phone": "string",
        "email": "string",
        "firstName": "string",
        "lastName": "string",
        "imageUrl": "string",
        "telegram": "string",
        "website": "string",
        "userRoleId": "string",
        "permissions": "string",
        "password": "string"
      \} & \{
        "phone": "string",
        "email": "string",
        "firstName": "string",
        "lastName": "string",
        "imageUrl": "string",
        "telegram": "string",
        "website": "string",
        "userRoleId": "string",
        "permissions": "string",
        "id": 0,
        "hash": "string",
        "salt": "string"
      \} \\
      \hline POST /api/User/login & Авторизоваться существующим пользователем User в системе. Авторизоваться -\- сохранить в клиентском приложение значение из поля accessToken & Body: \{
        "email": "string",
        "password": "string"
      \} & \{
        "phone": "string",
        "email": "string",
        "firstName": "string",
        "lastName": "string",
        "imageUrl": "string",
        "telegram": "string",
        "website": "string",
        "userRoleId": "string",
        "permissions": "string",
        "id": 0,
        "accessToken": "string"
      \} \\
\end{xltabular}

\renewcommand{\arraystretch}{1.0} % восстановление сетки

\subsection{Модульное тестирование разработанного приложения}

Класс Short является базовым классом, от которого наследуются почти все модели серверной части приложения. Модульный тест для класса Short из модели данных представлен на рисунке \ref{unitshort:image}.

\begin{figure}[ht]
\lstset{style=sharpc}
\begin{lstlisting}
    using System;
    using System.IO;
    using System.Math;
    using Microsoft.VisualStudio.TestTools.UnitTesting;
    
    namespace new_back
    {
      [TestClass]
      public class Short
      {
        public long Id { get; set; } = null;
    
        public string Name { get; set; } = null;
    
        public string Description { get; set; } = null;
      }
    
      [TestMethod]
      public void Short_Validate_Name() 
      { 
        var testValue = "Test Name";
        var testObject = new Short() { 
          Name = testValue
        };
        TestString(testObject.name, testValue); 
      } 
      
      [TestMethod]
      public void Short_Validate_Description() 
      { 
        var testValue = "Test description";
        var testObject = new Short() { 
          Description = testValue
        };
        TestString(testObject.name, testValue); 
      } 
      
      [TestMethod]
      public void Short_Validate_Id() 
      { 
        var testObject = new Short(); 
        long randomId = 0; 
        while (randomId == 0) 
        { 
          randomId = GenerateUniqueId(); 
        } 
        testObject.Id = randomId; 
        Assert.AreNotEqual(0, testObject.Id); 
        Assert.AreEqual(randomId, testObject.Id); 
      }
    
      private boolean TestString(string v1, string v2)
      {
        if (v1 == null || v2 == null)
          Assert.Fail();
          
        Assert.AreEqual(v1, v2);
      }
    
      public static int GenerateUniqueId()
      {
        Random random = new Random();
        return random.Next(1, 101);
      }
    }    
\end{lstlisting}  
\caption{Модульный тест класса Short}
\label{unitshort:image}
\end{figure}

\subsection{Системное тестирование разработанного web-сайта}

На рисунке \ref{main:image} представлена главная страница сайта «Русатом – Аддитивные технологии».
\newpage % при необходимости можно переносить рисунок на новую страницу
\begin{figure}[H] % H - рисунок обязательно здесь, или переносится, оставляя пустоту
\center{\includegraphics[width=1\linewidth]{main1}}
\center{\includegraphics[width=1\linewidth]{main2}}
\center{\includegraphics[width=1\linewidth]{main3}}
\caption{Главная страница сайта «Русатом – Аддитивные технологии»}
\label{main:image}
\end{figure}

На рисунке \ref{menu:image} представлен динамический вывод заголовков, включающий в себя искомые фразы при поиске фраз.

\begin{figure}[ht]
\center{\includegraphics[width=1\linewidth]{menu}}
\caption{Динамический вывод заголовков}
\label{menu:image}
\end{figure}

На рисунке \ref{enter:image} представлен ввод данных для публикации новости.

\begin{figure}[ht]
\center{\includegraphics[width=1\linewidth]{enter}}
\caption{Ввод данных для публикации очень-очень длинной, интересной и полезной новости}
\label{enter:image}
\end{figure}
