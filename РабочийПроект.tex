\section{Рабочий проект}

\subsection{Описание архитектуры приложения}

Архитектура приложения, реализованная в рамках текущего исследования, базируется на модели MVC (Model-View-Controller) \cite{kofman}. Этот паттерн был избран благодаря его выдающейся способности к эффективному разделению функциональных обязанностей между структурными компонентами системы, а также способствованию упрощению процессов разработки и тестирования.

Паттерн MVC реализован следующим образом:

\begin{enumerate}
    \item Модель (Model) и Контроллер (Controller): Обе эти компоненты были разработаны на платформе .NET, обеспечивая управление бизнес-логикой и обработку данных, а также связь между пользовательским интерфейсом и базой данных;
    \item Представление (View): Реализовано через фронтенд, использующий фреймворк Angular, отвечающий за визуализацию информации и интерактивное взаимодействие с пользователем.
\end{enumerate}

Ключевым аспектом в управлении данными является использование системы управления базами данных PostgreSQL \cite{kofman}. Этот выбор был обусловлен высокой надежностью PostgreSQL, её превосходной производительностью и гибкостью в обработке сложных запросов, что имеет критическое значение для эффективного функционирования бэкенда.

Применение архитектуры MVC принесло следующие ключевые преимущества:

\begin{itemize}
  \item   Четкое Разделение Обязанностей: Эффективное разграничение между пользовательским интерфейсом (Angular) и серверной логикой (.NET) значительно упрощает процесс разработки и последующей поддержки системы \cite{kofman};
  \item   Гибкость и Масштабируемость: Благодаря MVC, архитектура приложения легко адаптируется и масштабируется, что позволяет разработчикам модифицировать или расширять отдельные части системы без влияния на другие;
  \item   Упрощение Тестирования: Независимость компонентов архитектуры MVC облегчает процедуру тестирования, позволяя проводить её для каждого элемента в отдельности \cite{kofman}.
\end{itemize}

Итоговая архитектура, базирующаяся на MVC в сочетании с Angular для фронтенда, .NET для бэкенда и PostgreSQL для управления данными, обеспечивает создание надежной, гибкой и масштабируемой системы. Данная архитектура представляет собой устойчивую основу для разработки современных программных решений, способных соответствовать как нынешним, так и будущим техническим и функциональным требованиям \cite{kofman}.

\subsection{Описание REST API приложения}

Можно выделить следующий перечень HTTP-методов, использованных при разработке веб-приложения и доступных для использования клиентской частью приложения. Описание этих методов предоставлены в виде аблиц с уменьшенным межстрочным интервалом:
\begin{itemize}
    \item описания методов для работы с городами (таблица \ref{city:table});
    \item описание методов для работы с карточками продуктов (товаров и услуг) (таблица \ref{product:table});
    \item описание методов для работы с типами продуктов (товаров и услуг) (таблица \ref{type:table});
    \item описание методов для работы с данными пользователя (таблица \ref{data:table}).
\end{itemize}

\renewcommand{\arraystretch}{0.8} % уменьшение расстояний до сетки таблицы

\begin{xltabular}{\textwidth}{|X|p{2.5cm}|>{\setlength{\baselineskip}{0.7\baselineskip}}p{4.85cm}|>{\setlength{\baselineskip}{0.7\baselineskip}}p{4.85cm}|}
    \caption{Описания методов для работы с городами\label{city:table}}\\
    \hline \centrow \setlength{\baselineskip}{0.7\baselineskip} HTTP-метод & \centrow \setlength{\baselineskip}{0.7\baselineskip} Описание & \centrow Входные параметры & \centrow Пример JSON ответа \\
    \hline \centrow 1 & \centrow 2 & \centrow 3 & \centrow 4\\ \hline
    GET /api /City  & Список всех городов City в БД & Нет & [ \{ 
      id: 0, 
      name: <<string>>, 
      description: <<string>> 
        \} 
      ] \\
\end{xltabular}
    

\begin{xltabular}{\textwidth}{|X|p{2.5cm}|>{\setlength{\baselineskip}{0.7\baselineskip}}p{4.85cm}|>{\setlength{\baselineskip}{0.7\baselineskip}}p{4.85cm}|}
    \caption{Описание методов для работы с карточками продуктов (товаров и услуг)\label{product:table}}\\
    \hline \centrow \setlength{\baselineskip}{0.7\baselineskip} HTTP-метод & \centrow \setlength{\baselineskip}{0.7\baselineskip} Описание & \centrow Входные параметры & \centrow Пример JSON ответа \\
    \hline \centrow 1 & \centrow 2 & \centrow 3 & \centrow 4\\ \hline
    \endfirsthead
    \hline \centrow 1 & \centrow 2 & \centrow 3 & \centrow 4\\ \hline
    \finishhead
    GET /api /Product  & Список всех карточек Product & Нет & [
        \{
      id: 0,
      name: <<string>>,
      description: <<string>>,
      imageUrl: <<string>>,
      price: 0,
      createdByUserId: 0,
      cityId: 0,
      productGroupId: 0,
      productStatusId: 0,
      createdAt: <<2024-01-20>>,
      publishedAt: <<2024-01-20>>,
      expiredAt: <<2024-01-20>>,
      priorityId: 0,
      priorityStartedAt: <<2024-01-20>>,
      priorityExpiredAt: <<2024-01-20>>,
      createdByUser: \{
        phone: <<string>>,
        email: <<string>>,
        firstName: <<string>>,
        lastName: <<string>>,
        imageUrl: <<string>>,
        telegram: <<string>>,
        website: <<string>>,
        userRoleId: <<string>>,
        permissions: <<string>>,
        id: 0
          \}
        \}
      ]\\
\hline POST /api /Product  & Создать новый Product & Body: \{
id: 0,
name: <<string>>,
description: <<string>>,
imageUrl: <<string>>,
price: 0,
createdByUserId: 0,
cityId: 0,
productGroupId: 0,
productStatusId: 0,
createdAt: <<2024-01-20>>,
publishedAt: <<2024-01-20>>,
expiredAt: <<2024-01-20>>,
priorityId: 0,
priorityStartedAt: <<2024-01-20>>,
priorityExpiredAt: <<2024-01-20>>,
createdByUser: \{
  phone: <<string>>,
  email: <<string>>,
  firstName: <<string>>,
  lastName: <<string>>,
  imageUrl: <<string>>,
  telegram: <<string>>,
  website: <<string>>,
  userRoleId: <<string>>,
  permissions: <<string>>,
  id: 0
    \}
  \} & \{
id: 0,
name: <<string>>,
description: <<string>>,
imageUrl: <<string>>,
price: 0,
createdByUserId: 0,
cityId: 0,
productGroupId: 0,
productStatusId: 0,
createdAt: <<2024-01-20>>,
publishedAt: <<2024-01-20>>,
expiredAt: <<2024-01-20>>,
priorityId: 0,
priorityStartedAt: <<2024-01-20>>,
priorityExpiredAt: <<2024-01-20>>
  \}\\
\hline GET /api /Product /\{id\} & Получить выбранный Product по его id & Нет & \{
id: 0,
name: <<string>>,
description: <<string>>,
imageUrl: <<string>>,
price: 0,
createdByUserId: 0,
cityId: 0,
productGroupId: 0,
productStatusId: 0,
createdAt: <<2024-01-20>>,
publishedAt: <<2024-01-20>>,
expiredAt: <<2024-01-20>>,
priorityId: 0,
priorityStartedAt: <<2024-01-20>>,
priorityExpiredAt: <<2024-01-20>>
  \}\\
\hline PUT /api /Product /\{id\} & Обновить существующий Product & Body: \{
id: 0,
name: <<string>>,
description: <<string>>,
imageUrl: <<string>>,
price: 0,
createdByUserId: 0,
cityId: 0,
productGroupId: 0,
productStatusId: 0,
createdAt: <<2024-01-20>>,
publishedAt: <<2024-01-20>>,
expiredAt: <<2024-01-20>>,
priorityId: 0,
priorityStartedAt: <<2024-01-20>>,
priorityExpiredAt: <<2024-01-20>>,
createdByUser: \{
  phone: <<string>>,
  email: <<string>>,
  firstName: <<string>>,
  lastName: <<string>>,
  imageUrl: <<string>>,
  telegram: <<string>>,
  website: <<string>>,
  userRoleId: <<string>>,
  permissions: <<string>>,
  id: 0
    \}
  \} & \{
id: 0,
name: <<string>>,
description: <<string>>,
imageUrl: <<string>>,
price: 0,
createdByUserId: 0,
cityId: 0,
productGroupId: 0,
productStatusId: 0,
createdAt: <<2024-01-20>>,
publishedAt: <<2024-01-20>>,
expiredAt: <<2024-01-20>>,
priorityId: 0,
priorityStartedAt: <<2024-01-20>>,
priorityExpiredAt: <<2024-01-20>>
  \}\\
\hline DELETE /api /Product /\{id\} & Удалить Product по его id & Нет & Нет\\
\hline GET /api /Product /filter & Получить отфильтрованный список существующих продуктов & Query parameters: cityId: integer, productGroupId: integer, publishedAtFrom: string, expiredAtTo: string. & [
    \{
  id: 0,
  name: <<string>>,
  description: <<string>>,
  imageUrl: <<string>>,
  price: 0,
  createdByUserId: 0,
  cityId: 0,
  productGroupId: 0,
  productStatusId: 0,
  createdAt: <<2024-01-20>>,
  publishedAt: <<2024-01-20>>,
  expiredAt: <<2024-01-20>>,
  priorityId: 0,
  priorityStartedAt: <<2024-01-20>>,
  priorityExpiredAt: <<2024-01-20>>
    \}
  ]\\
\end{xltabular}

\begin{xltabular}{\textwidth}{|X|p{2.5cm}|>{\setlength{\baselineskip}{0.7\baselineskip}}p{4.85cm}|>{\setlength{\baselineskip}{0.7\baselineskip}}p{4.85cm}|}
    \caption{Описание методов для работы с типами продуктов (товаров и услуг)\label{type:table}}\\
    \hline \centrow \setlength{\baselineskip}{0.7\baselineskip} HTTP-метод & \centrow \setlength{\baselineskip}{0.7\baselineskip} Описание & \centrow Входные параметры & \centrow JSON ответ \\
    \hline \centrow 1 & \centrow 2 & \centrow 3 & \centrow 4\\ \hline
    \endfirsthead
    \hline \centrow 1 & \centrow 2 & \centrow 3 & \centrow 4\\ \hline
    \finishhead
    GET /api /ProductGtoup  & Список всех типов продуктов (товаров и услуг) ProductGroup в БД & Нет & [
        \{
      id: 0,
      name: <<string>>,
      description: <<string>>,
      type: 0,
      childrenProductGroupIds: [
            0
          ],
      parentProductGroupId: 0
        \}
      ]\\
      \hline POST /api /ProductGtoup  & Создать новый ProductGtoup & Body: [
        \{
      id: 0,
      name: <<string>>,
      description: <<string>>,
      type: 0,
      childrenProductGroupIds: [
            0
          ],
      parentProductGroupId: 0
        \}
      ] & [
        \{
      id: 0,
      name: <<string>>,
      description: <<string>>,
      type: 0,
      childrenProductGroupIds: [
            0
          ],
      parentProductGroupId: 0
        \}
      ] \\
\end{xltabular}

\begin{xltabular}{\textwidth}{|X|p{4.85cm}|>{\setlength{\baselineskip}{0.7\baselineskip}}p{2.5cm}|>{\setlength{\baselineskip}{0.7\baselineskip}}p{4.85cm}|}
    \caption{Описание методов для работы с данными пользователя\label{data:table}}\\
    \hline \centrow \setlength{\baselineskip}{0.7\baselineskip} HTTP-метод & \centrow \setlength{\baselineskip}{0.7\baselineskip} Описание & \centrow Входные параметры & \centrow JSON ответ \\
    \hline \centrow 1 & \centrow 2 & \centrow 3 & \centrow 4\\ \hline
    \endfirsthead
    \hline \centrow 1 & \centrow 2 & \centrow 3 & \centrow 4\\ \hline
    \finishhead
    GET /api /User /{id} & Получить данные существующего пользователя User по его id & Нет & \{
    phone: <<string>>,
    email: <<string>>,
    firstName: <<string>>,
    lastName: <<string>>,
    imageUrl: <<string>>,
    telegram: <<string>>,
    website: <<string>>,
    userRoleId: <<string>>,
    permissions: <<string>>,
    id: 0
      \}\\
      \hline POST /api /User /create & Создать нового пользователя User & Body: \{
    phone: <<string>>,
    email: <<string>>,
    firstName: <<string>>,
    lastName: <<string>>,
    imageUrl: <<string>>,
    telegram: <<string>>,
    website: <<string>>,
    userRoleId: <<string>>,
    permissions: <<string>>,
    password: <<string>>
      \} & \{
    phone: <<string>>,
    email: <<string>>,
    firstName: <<string>>,
    lastName: <<string>>,
    imageUrl: <<string>>,
    telegram: <<string>>,
    website: <<string>>,
    userRoleId: <<string>>,
    permissions: <<string>>,
    id: 0,
    hash: <<string>>,
    salt: <<string>>
      \} \\
      \hline POST /api /User /login & Авторизоваться существующим пользователем User в системе. Авторизоваться -\- сохранить в клиентском приложение значение из поля accessToken & Body: \{
    email: <<string>>,
    password: <<string>>
      \} & \{
    phone: <<string>>,
    email: <<string>>,
    firstName: <<string>>,
    lastName: <<string>>,
    imageUrl: <<string>>,
    telegram: <<string>>,
    website: <<string>>,
    userRoleId: <<string>>,
    permissions: <<string>>,
    id: 0,
    accessToken: <<string>>
      \} \\
\end{xltabular}

\renewcommand{\arraystretch}{1.0} % восстановление сетки

\subsection{Модульное тестирование разработанного приложения}


\subsubsection{Структура тестового проекта}
В рамках тестирования программного обеспечения был разработан тестовый проект, структура которого представлена на рисунке \ref{testproject_structure:image}. Данный проект включает в себя модульные тесты для различных компонентов системы. Тестирование моделей и контроллеров позволяет обеспечить надежность и корректность работы программного обеспечения.
В тестовый проект добавлена возможность использовать пространство имен из основного проекта с помощью команды: \texttt{dotnet add reference ../new\textunderscore back/new\textunderscore back.csproj}.
Тестовый проект включает в себя следующие ключевые элементы:

\begin{itemize}
  \item Models: Классы, которые представляют модели данных, используемые в тестах;
  \item Shared: Вспомогательные классы и утилиты, используемые в различных тестах;
  \item TestResults: Результаты выполнения тестов;
  \item UnitTestShort.cs и UnitTestUserController.cs: Наборы модульных тестов для проверки соответствующих классов и контроллеров.
\end{itemize}

\begin{figure}[ht]
\centering
\includegraphics[width=0.5\textwidth]{testproject_structure}
\caption{Структура тестового проекта}
\label{testproject_structure:image}
\end{figure}

Содержимое файла ShortTest.cs представлено на рисунке \ref{st:image}.

\begin{figure}[!ht]
\lstset{style=sharpc}
\begin{lstlisting}
namespace TestProject.Models
{
    [TestClass]
    public class ShortTest
    {
        public long Id { get; set; }
        public string? Name { get; set; }
        public string? Description { get; set; } = null;
    }
}
\end{lstlisting}
\caption{Содержимое файла ShortTest.cs}
\label{st:image}
\end{figure}

Содержимое файла Utils.cs представлено на рисунке \ref{ut:image}.

\begin{figure}[!ht]
\lstset{style=sharpc}
\begin{lstlisting}
namespace TestProject.Shared
{
    public class Utils
    {
        public static void TestString(string v1, string v2)
        {
            if (v1 == null || v2 == null)
            {
                Assert.Fail();
            }

            Assert.AreEqual(v1, v2);
        }

        public static int GenerateUniqueId()
        {
            Random random = new Random();
            return random.Next(1, 101);
        }
    }
}
\end{lstlisting}
\caption{Содержимое файла Utils.cs}
\label{ut:image}
\end{figure}

Содержимое файла Usings.cs представлено на рисунке \ref{usings:image}.

\begin{figure}[!ht]
\lstset{style=sharpc}
\begin{lstlisting}
global using Microsoft.VisualStudio.TestTools.UnitTesting;
global using System.IO;
global using static System.Math;
global using System.Web;
global using Moq;
global using Microsoft.AspNetCore.Mvc;
\end{lstlisting}
\caption{Содержимое файла Usings.cs}
\label{usings:image}
\end{figure}

Модульные тесты для класса Short, который является базовым классом для большинства моделей в приложении, позволяют проверить корректность логики и поведения этого компонента и исключить возникновение проблем в этом классе и его потомков. Тесты охватывают такие аспекты, как валидация имен и описаний, а также уникальность идентификаторов объектов.
Модульные тесты для класса Short представлены на рисунке \ref{unitshort1:image}.

\begin{figure}[ht]
\lstset{style=sharpc}
\begin{lstlisting}
using TestProject.Shared;
using TestProject.Models;

namespace TestProject
{
    [TestClass]
    public class UnitTestShort
    {
        [TestMethod]
        public void Short_Validate_Name()
        {
            var testValue = "Test Name";
            var testObject = new ShortTest() {
                Name = testValue
            };
            Utils.TestString(testObject.Name, testValue);
        }

        [TestMethod]
        public void Short_Validate_Description()
        {
            var testValue = "Test description";
            var testObject = new ShortTest() {
                Description = testValue
            };
            Utils.TestString(testObject.Description, testValue);
        }

        [TestMethod]
        public void Short_Validate_Id()
        {
            var testObject = new ShortTest();
            long randomId = 0;
            while (randomId == 0)
            {
                randomId = Utils.GenerateUniqueId();
            }
            testObject.Id = randomId;
            Assert.AreNotEqual(0, testObject.Id);
            Assert.AreEqual(randomId, testObject.Id);
        }
    }
}  
\end{lstlisting}
\caption{Модульный тест класса Short}
\label{unitshort1:image}
\end{figure}

Для контроллера UserController, который обрабатывает пользовательские запросы, были разработаны тесты, имитирующие взаимодействие с репозиторием данных. Это позволяет тестировать контроллер в изоляции от реальной базы данных, что упрощает процесс разработки и повышает качество кода.
Модульные тесты для UserController представлены в двух частях на рисунках \ref{unitcontr1:image} и \ref{unitcontr2:image}.

\begin{figure}[ht]
\lstset{style=sharpc}
\begin{lstlisting}
global using new_back.Controllers;
global using new_back.Infrastructure;
global using new_back.Models;

namespace TestProject
{
    [TestClass]
    public class UserControllerTests
    {
        private Mock<IUserRepository> _mockRepository;
        private UserController _controller;

        [TestInitialize]
        public void SetUp()
        {
            _mockRepository = new Mock<IUserRepository>();
            _controller = new UserController(_mockRepository.Object);
        }

        [TestMethod]
        public void GetById_UserExists_ReturnsUserView()
        {
            long userId = 1;
            var mockUser = new User { Id = userId, Email = "test@example.com" };
            _mockRepository.Setup(repo => repo.GetById(userId)).Returns(mockUser);

            var result = _controller.GetById(userId);

            Assert.IsNotNull(result);
            Assert.IsInstanceOfType(result.Result, typeof(OkObjectResult));

            var okResult = result.Result as OkObjectResult;
            Assert.IsNotNull(okResult);

            var userViewResult = okResult.Value as UserView;
            Assert.IsNotNull(userViewResult);
            Assert.AreEqual(userId, userViewResult.Id);
            Assert.AreEqual("test@example.com", userViewResult.Email);
        }
\end{lstlisting}
\caption{Модульный тест класса UserController часть 1}
\label{unitcontr1:image}
\end{figure}


\begin{figure}[ht]
\lstset{style=sharpc}
\begin{lstlisting}
    [TestMethod]
    public void GetById_UserDoesNotExist_ReturnsNotFound()
    {
        _mockRepository.Setup(repo => repo.GetById(It.IsAny<long>())).Returns((User)null);

        var result = _controller.GetById(1);

        Assert.IsInstanceOfType(result.Result, typeof(NotFoundResult));
    }
  }
}
\end{lstlisting}
\caption{Модульный тест класса UserController часть 2}
\label{unitcontr2:image}
\end{figure}

Разработка тестового проекта является ключевым компонентом в цикле обеспечения качества программного продукта. Она демонстрирует эффективность тестирования и его способность гарантировать надежность функционирования системы. На рисунке \ref{testresult:image} представлены результаты выполнения модульных тестов, подтверждающие успешную проверку всех компонентов системы, что свидетельствует об отсутствии критических ошибок и готовности продукта к дальнейшим этапам разработки и внедрения.

\newpage % при необходимости можно переносить рисунок на новую страницу
\begin{figure}[H] % H - рисунок обязательно здесь, или переносится, оставляя пустоту
\centering
\includegraphics[width=0.8\textwidth]{testresult}
\caption{Результаты выполнения модульных тестов}
\label{testresult:image}
\end{figure}

\subsection{Системное тестирование разработанного веб-приложения}

Каждая страница веб-приложения включает в себя такие элементы интерфейса как шапка сайта, боковое меню и основное содержимое страницы.

Шапка сайта состоит из:
\begin{itemize}
    \item кнопка для раскрытия или закрытия бокового меню в зависимости от его текущего состояния;
    \item логотип;
    \item кнопки для перехода на страницу входа в систему и регистрации или имя и фамилию пользователя если он авторизован в системе и кнопку выхода пользователя из системы.
\end{itemize}

Боковое меню включает в себя ссылки на различные страницы сайта.
Основное содержимое страницы меняется в зависимости от выбранного раздела сайта и включает в себя заголовок страницы и ее содержимое. 

Главная страница включает в себя список карточек продуктов (товаров и услуг), шапку сайта и боковое меню. Пользователь может просматривать карточки продуктов (товаров и услуг), а также раскрывать при нажатии на кнопку <<Подробнее>> раскрывать блок с детальной информацией о товаре или услуге.
На рисунке \ref{test-front1:image} представлена главная страница веб-приложения для неавторизованного пользователя с открытым боковым меню. Первая карточка с услугой <<Кошки в аренду>> раскрыта для просмотра детальной информации.

\begin{figure}[H] % H - рисунок обязательно здесь, или переносится, оставляя пустоту
\center{\includegraphics[width=1\linewidth]{test-front1}}
\caption{Главная страница веб-приложения}
\label{test-front1:image}
\end{figure}

На рисунке \ref{test-front2:image} представлена страница регистрации пользователя-поставщика услуг. На рисунке \ref{test-front2:image} также изображено боковым меню в закрытом состоянии.

\newpage % при необходимости можно переносить рисунок на новую страницу
\begin{figure}[H] % H - рисунок обязательно здесь, или переносится, оставляя пустоту
\center{\includegraphics[width=1\linewidth]{test-front2}}
\caption{Страница регистрации пользователя-поставщика услуг}
\label{test-front2:image}
\end{figure}

После регистрации пользователь автоматически авторизуется в системе и перенаправляется на главную страницу приложения где он может просматривать карточки продуктов (товаров и услуг), а также создавать свои карточки. На рисунке \ref{test-front3:image} представлена страница создания карточки. На рисунке \ref{test-front3:image} также изображено состояние шапки сайта для авторизованного пользователя.

\newpage % при необходимости можно переносить рисунок на новую страницу
\begin{figure}[H] % H - рисунок обязательно здесь, или переносится, оставляя пустоту
\center{\includegraphics[width=1\linewidth]{test-front3}}
\caption{Страница регистрации создания карточки}
\label{test-front3:image}
\end{figure}

Созданная карточка отображается на главной странице веб-приложения. На рисунке \ref{test-front4:image} представлена главная страница веб-приложения для авторизованного пользователя с открытым боковым меню. Новая карточка с названием <<Тестовое название>> содержит кнопку для редактирования при нажатии на которую пользователь перенаправляется на страницу редактирования этой карточки. Новая карточка также содержит кнопку удаления. На рисунке \ref{test-front4:image} изображена обновленная главная страница веб-приложения.

\newpage % при необходимости можно переносить рисунок на новую страницу
\begin{figure}[H] % H - рисунок обязательно здесь, или переносится, оставляя пустоту
\center{\includegraphics[width=1\linewidth]{test-front4}}
\caption{Обновленная главная страница веб-приложения}
\label{test-front4:image}
\end{figure}

На рисунке \ref{test-front5:image} изображена страница редактирования карточки с новым выбранным изображением товара в поле <<Изображение>>. На рисунке \ref{test-front6:image} изображена страница редактирования карточки после нажатия на кнопку <<Сохранить>>.

\newpage % при необходимости можно переносить рисунок на новую страницу
\begin{figure}[H] % H - рисунок обязательно здесь, или переносится, оставляя пустоту
\center{\includegraphics[width=1\linewidth]{test-front5}}
\caption{Страница редактирования карточки}
\label{test-front5:image}
\end{figure}

\newpage % при необходимости можно переносить рисунок на новую страницу
\begin{figure}[H] % H - рисунок обязательно здесь, или переносится, оставляя пустоту
\center{\includegraphics[width=1\linewidth]{test-front6}}
\caption{Обновленная страница редактирования карточки}
\label{test-front6:image}
\end{figure}

На рисунке \ref{test-front7:image} изображена страница поиска с предустановлеными фильтрами. На рисунке \ref{test-front7:image} изображена страница результатами поиска с заданными на странице поиска фильтрами.

\begin{figure}[H] % H - рисунок обязательно здесь, или переносится, оставляя пустоту
\center{\includegraphics[width=1\linewidth]{test-front7}}
\caption{Страница поиска}
\label{test-front7:image}
\end{figure}

\begin{figure}[H] % H - рисунок обязательно здесь, или переносится, оставляя пустоту
\center{\includegraphics[width=1\linewidth]{test-front8}}
\caption{Страница с результатами поиска}
\label{test-front8:image}
\end{figure}
