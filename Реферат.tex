\abstract{РЕФЕРАТ}

Объем работы равен \formbytotal{lastpage}{страниц}{е}{ам}{ам}. Работа содержит \formbytotal{figurecnt}{иллюстраци}{ю}{и}{й}, \formbytotal{tablecnt}{таблиц}{у}{ы}{}, \arabic{bibcount} библиографических источников и \formbytotal{числоПлакатов}{лист}{}{а}{ов} графического материала. Количество приложений – 3. Графический материал представлен в приложении А. Фрагменты исходного кода серверной части приложения представлены в приложении Б. Фрагменты исходного кода клиентской части приложения представлены в приложении В.

Перечень ключевых слов: веб-приложение, домашние животные, социальная сеть, социомаркет.

Объектом разработки является веб-приложение <<Социомаркет для владельцев домашних животных>>, представляющее собой платформу, на которой владельцы домашних животных могут находить и предлагать различные услуги и товары.

Целью работы является создания веб-приложения с интуитивно понятным интерфейсом и полезными функциями.

Серверная часть системы разработана на .NET Core, а клиентская часть на Angular. База данных системы построена на PostgreSQL, что обеспечивает её надежность и эффективность.

Для работы приложения на стороне пользователя необходимо мобильное устройство или компьютер с доступом в интернет с установленным браузером. Приложение оптимизировано для работы как в современных так и в более старых версиях браузеров и не требует значительных ресурсов аппаратного обеспечения.

Для размещения программы на стороне сервера необходим компьютер с операционной системой Windows 10, процессор с тактовой частотой не ниже 1.6 ГГц, ОЗУ 1024 Мб. Для обеспечения стабильной работы базы данных необходимо место на жестком диске 1 ГБ или больше. Для работы серверной части приложения также потребуется установка программной платформы .NET 6.0, а для клиентской части -\- Node.js v12.11.0.

Состояние системы: веб-приложение находится на стадии внедрения и доступно для использования всеми заинтересованными сторонами.

\selectlanguage{english}
\abstract{ABSTRACT}
  
The volume of work is \formbytotal{lastpage}{page}{}{s}{s}. The work contains \formbytotal{figurecnt}{illustration}{}{s}{s}, \formbytotal{tablecnt}{table}{}{s}{s}, \arabic{bibcount} bibliographic sources and \formbytotal{числоПлакатов}{sheet}{}{s}{s} of graphic material. The number of applications is 3. The graphic material is presented in annex A. Fragments of the source code of the server part of the application are presented in Appendix B. Fragments of the source code of the client part of the application are presented in Appendix C.

List of keywords: web application, pets, social network, social marketplace.

The object of development is the web application <<Sociomarket for pet owners>>, which is a platform on which pet owners can find and offer various services and products.

The goal of the work is to create a convenient and functional platform.

The server part of the system is developed based on .NET Core, and the client part is based on Angular. The system database is built on PostgreSQL, which ensures its reliability and efficiency.

To operate the application on the user side, you need a mobile device or computer with Internet access and a browser installed. The application is optimized to work in both modern and older versions of browsers and does not require significant hardware resources.

To host the program on the server side, you need a computer with the Windows 10 operating system, a processor with a clock frequency of at least 1.6 GHz, and 1024 MB of RAM. To ensure stable operation of the database, a hard disk space of 1 GB or more is required. To operate the server part of the application, you will also need to install the .NET 6.0 software platform, and for the client part -\- Node.js v12.11.0.

System Status: The web application is in the implementation stage and is available for use by all interested parties. It provides a unique opportunity for pet owners to interact and exchange services and goods in a convenient and safe environment.

\selectlanguage{russian}
