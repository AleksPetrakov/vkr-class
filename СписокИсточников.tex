% \addcontentsline{toc}{section}{СПИСОК ИСПОЛЬЗОВАННЫХ ИСТОЧНИКОВ}
%
% \begin{thebibliography}{9}
%
%     \bibitem{javascript} Фримен, А. Практикум по программированию на JavaScript / А. Фримен. – Москва~: Вильямс, 2013. – 960 с. – ISBN 978-5-8459-1799-7. – Текст~: непосредственный.
%     \bibitem{php} Бретт, М. PHP и MySQL. Исчерпывающее руководство / М. Бретт. – Санкт-Петербург : Питер, 2016. – 544 с. – ISBN 978-5-496-01049-8. – Текст~: непосредственный.
%     \bibitem{css} Веру, Л. Секреты CSS. Идеальные решения ежедневных задач / Л. Веру. – Санкт-Петербург : Питер, 2016. – 336 с. – ISBN 978-5-496-02082-4. – Текст~: непосредственный.
%     \bibitem{mysql}	Гизберт, Д. PHP и MySQL / Д. Гизберт. – Москва~: НТ Пресс, 2013. – 320 с. – ISBN 978-5-477-01174-2. – Текст~: непосредственный.
% 	\bibitem{html5}	Голдстайн, А. HTML5 и CSS3 для всех / А. Голдстайн, Л. Лазарис, Э. Уэйл. – Москва~: Вильямс, 2012. – 368 с. – ISBN 978-5-699-57580-0. – Текст~: непосредственный.
% 	\bibitem{htmlcss}	Дэкетт, Д. HTML и CSS. Разработка и создание веб-сайтов / Д. Дэкетт. – Москва~: Эксмо, 2014. – 480 с. – ISBN 978-5-699-64193-2. – Текст~: непосредственный.
% 	\bibitem{bigbook}	Макфарланд, Д. Большая книга CSS / Д. Макфарланд. – Санкт-Петербург : Питер, 2012. – 560 с. – ISBN 978-5-496-02080-0. – Текст~: непосредственный.
% 	\bibitem{uchiru}	Лоусон, Б. Изучаем HTML5. Библиотека специалиста / Б. Лоусон, Р. Шарп. – Санкт-Петербург : Питер, 2013 – 286 с. – ISBN 978-5-459-01156-2. – Текст~: непосредственный.
% 	\bibitem{chaynik}	Титтел, Э. HTML5 и CSS3 для чайников / Э. Титтел, К. Минник. – Москва~: Вильямс, 2016 – 400 с. – ISBN 978-1-118-65720-1. – Текст~: непосредственный.
% 	\bibitem{22}	Титтел, Э. HTML5 и CSS3 для чайников / Э. Титтел, К. Минник. – Москва~: Вильямс, 2016 – 400 с. – ISBN 978-1-118-65720-1. – Текст~: непосредственный.
% 	\bibitem{1231}	Титтел, Э. HTML5 и CSS3 для чайников / Э. Титтел, К. Минник. – Москва~: Вильямс, 2016 – 400 с. – ISBN 978-1-118-65720-1. – Текст~: непосредственный.
% 	\bibitem{sdf}	Титтел, Э. HTML5 и CSS3 для чайников / Э. Титтел, К. Минник. – Москва~: Вильямс, 2016 – 400 с. – ISBN 978-1-118-65720-1. – Текст~: непосредственный.
% 	\bibitem{servsssds}	Титтел, Э. HTML5 и CSS3 для чайников / Э. Титтел, К. Минник. – Москва~: Вильямс, 2016 – 400 с. – ISBN 978-1-118-65720-1. – Текст~: непосредственный.
% \end{thebibliography}
\addcontentsline{toc}{section}{СПИСОК ИСПОЛЬЗОВАННЫХ ИСТОЧНИКОВ}

\begin{thebibliography}{14}

    \bibitem{yordon} Йордон, Э. Путь камикадзе. Как разработчику программного обеспечения выжить в безнадежном проекте / Эдвард Йордон. – Москва : Лори, 2019. – 256 с.

    \bibitem{averin} Аверин, А.В. Стандартизация и регламентация государственных услуг: российский и зарубежный опыт / А.В. Аверин. – Москва : Урал, 2015. – С. 721-727.

    \bibitem{grinchenko} Гринченко, Н.Н. Проектирование баз данных. СУБД Microsoft Ассе: Учебное пособие для вузов / Н.Н. Гринченко и др. – Москва : РиС, 2013. – 240 с.

    \bibitem{postgresql} Postgresql.org -\- Документация базы данных : сайт. - URL: https://www.postgresql.org/docs/ (дата обращения: 20.05.2022). - Текст : электронный.

    \bibitem{akhrameeva} Ахрамеева, О.В. Российское сервисное государство: теоретические основы и государственная стратегия обеспечения частных и публичных интересов / О.В. Ахрамеева. – Москва : Урал, 2014. – С. 1-28.

    \bibitem{makni} Макни, Дж.К. Проектирование серверной инфраструктуры баз данных Microsoft SQL Server 2005 / Дж.К. Макни. – Москва : Русская редакция, 2008. – 560 с.

    \bibitem{cssspecs} W3C Specifications -\- спецификация CSS: сайт. – URL: https://www.w3.org/Style/CSS/specs.en.html (дата обращения: 20.12.2023). – Текст: электронный.

    \bibitem{htmlbook} HTMLBOOK -\- спецификация НТМL: сайт. - URL: http://htmlbook.ru/html (дата обращения: 10.12.2023). - Текст : электронный.

 	\bibitem{html5}	Голдстайн, А. HTML5 и CSS3 для всех / А. Голдстайн, Л. Лазарис, Э. Уэйл. – Москва~: Вильямс, 2012. – 368 с. – ISBN 978-5-699-57580-0. – Текст~: непосредственный.

    \bibitem{boyko} Бойко И. Объектно-ориентированные СУБД / И. Бойко. – Киев : Высшая школа, 2014.

    \bibitem{kumskova} Кумскова, И.А. Базы данных. Учебник для ССУЗов / И.А. Кумскова. – Москва : КноРус, 2012. – 488 с.

    \bibitem{mark_price} Марк Прайс. C\# 9 и .NET 5. Разработка и оптимизация / Марк Прайс. – Санкт-Петербург : Питер, 2021. – 832 с. – ISBN 978-5-4461-2921-8. – Текст: непосредственный.

    \bibitem{msdn} MSDN -\- сеть разработчиков Microsoft [Электронный ресурс]. URL: https://msdn.microsoft.com.

    \bibitem{stackoverflow} StackOverflow -\- сайт вопросов и ответов для программистов [Электронный ресурс]. URL: https://ru.stackoverflow.com/.

    \bibitem{freeman} Фримен, Э. Изучаем Angular: Руководство по созданию приложений на Angular / Эрик Фримен, Элизабет Робсон. – Москва: Эксмо, 2018. – 720 с. – ISBN 978-5-699-98165-6.

    \bibitem{troelsen} Троелсен, Э. Язык программирования C\# 7 и платформы .NET и .NET Core / Эндрю Троелсен, Филип Джепикс. – Москва: Вильямс, 2018. – 1456 с. – ISBN 978-5-8459-2114-9.

    \bibitem{freedman} Фридман, Д. Методы исследования рынков / Дэвид Фридман, Сэм Либерман. – Москва: Альпина Паблишер, 2016. – 560 с. – ISBN 978-5-9614-5403-5.

    \bibitem{riccardi} Риккарди, Дж. ASP.NET Core 6: Современное развитие веб-приложений на C\# / Джейсон Риккарди. – Москва: ДМК Пресс, 2022. – 512 с. – ISBN 978-5-97060-837-9.

    \bibitem{kofman} Кофман, Л. Паттерны проектирования для эффективной работы с базами данных: EF Core / Леонид Кофман. – Москва: Рид Групп, 2020. – 336 с. – ISBN 978-5-496-03125-3.

    \bibitem{market} Маркетинговые исследования рынка. Учебное пособие. / Б.И. Герасимов, Н.Н. Мозгов. – Москва : Издательство: Форум, 2014. – 336 с. ISBN978-5-91134-811-3. – Текст : непосредственный.
\end{thebibliography}
